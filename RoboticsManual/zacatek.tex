\section{Cíl: první robot}


Tento text je určen pro začátečníky a~mírně pokročilé v~oblasti stavby autonomních robotů -- převážně středoškoláky v~prvním ročníku, kteří se pokoušejí postavit svého prvního \hyperref[autonomni]{autonomního}
robota, nejčastěji pro nějakou \hyperref[prehled_soutezi]{soutěž}.

Protože byl sepsán na pracovištích \href{http://www.helceletka.cz/robotarna}{Robotárna}\footnote{pobočka DDM Helceletka Brno}  a 
 \href{http://www.sokolska.cz/}{SPŠ Sokolská}\footnote{Střední průmyslová škola a Vyšší odborná škola Brno, Sokolská, příspěvková organizace},
 jsou některé části určené především členům jejich kroužků. Ale většina textu je použitelná všeobecně.

Každý robot se musí 
\begin{itemize} 	% \itemsep = 0ex   
\item vyrobit (mechanická část -- konstrukce)
\item osadit elektronikou, motory a~pod.
\item naprogramovat 
\end{itemize}

Přitom můžou nastat zhruba dvě situace:

\begin{itemize} % \style O
\item  Nemám žádné znalosti a~zkušenosti $\rightarrow$ doporučený 
postup je začít ve školním kroužku nebo samostatně stavět a~programovat 
robota z~\href{https://www.lego.com/cs-cz/mindstorms}{LEGO MINDSTORMS} -- viz kapitola \ref{lego:stavba}. 
Je to daleko nejjednodušší a~nejrychlejší cesta, omezení jste cenou stavebnice a~jejími možnostmi.
\item  Už jsem někdy něco naprogramoval, zapojil nebo postavil a~chci jít dál $\rightarrow$ potom je pro vás určen tento text.
Jednotlivé kapitoly se věnují různým oblastem, které je postupně potřeba zvládnout (nebo jimi pověřit jiného člena týmu) s~důrazem na začátečnické problémy.
\end{itemize}

\section{Týmy pro stavbu robotů}

Už bylo zmíněno, že stavba robotů zahrnuje tři propojené, 
ale relativně nezávislé okruhy: návrh a~výrobu mechanické konstrukce, 
návrh a~zapojení elektroniky a~programování.
Proto je dobré roboty stavět v~týmech, kde se jednotliví členové zaměřují na tyto oblasti a~navzájem se doplňují.
Navíc každý tým potřebuje řadu pomocných činností (nákup součástek, vyhledávání údajů na internetu a~pod.).
Je dobré mít proto v~každém týmu ještě pomocníka, který podporuje ostatní a~umožňuje jim soustředit se na jejich hlavní úkoly.

Úplně ideální potom je, když každou funkci v~týmu zastávají dva lidé, takže se mohou vzájemně zastupovat.
Tým by potom měl celkem osm členů -- dva mechaniky, dva elektroniky, dva programátory a~dva pomocníky.
To se ale v~praxi téměř nikdy nepodaří.
Často nastane právě opačný případ, kdy tým má pouze dva nebo tři členy, kteří se o~všechny činnosti musí nějak podělit.

V~každém případě ale platí, že je výhoda, pokud lidé v~týmu znají i~věci mimo jejich \uv{hlavní obor}, tj. když například programátor zná základy elektroniky.

\section{Postup návrhu robota}

\begin{enumerate} % \style n
\item  vybrat \hyperref[prehled_soutezi]{soutěž}, které se chcete zúčastnit nebo mít jiný cíl, proč stavět robota 
\item  stanovit, co by měl robot splnit -- přibližně 
\item  sepsat, co by měl robot splnit -- podrobně (klidně i~více stran A4), z~toho vyplyne
\item  zjištění, jaké senzory a~pohony robot potřebuje a~kde budou na robotovi umístěné
\item  návrh první konstrukce robota včetně umístění senzorů a~pohonů \label{robot:postup:navrh}
\item  zprovoznění toho všeho -- {\bf hlavní cíl tohoto textu }
\end{enumerate}


Začátečníci obvykle tuto posloupnost nedodrží, začnou bodem \ref{robot:postup:navrh} a~pak staví mechaniku pro 
3 a~více verzí robota (někdy tak odlišných, že už se vlastně jedná o různé roboty). Až potom zjistí, že je to dost práce navíc.
A~taky dost času navíc, který potom před soutěží chybí. 

\section{Co potřebujete na začátku}

{\bf Hardware: }

\begin{itemize} 
\item  notebook nebo počítač s~operačním systémem Windows 7 a~novějším nebo s operačním systémem Linux (pro starší počítače např. aktuální distribuci Xubuntu, Lubuntu, Mint). 
Podrobnější informace o doporučených parametrech počítačů jsou v kapitole \ref{pocitac}.
\item \hyperref[ridici_desky]{řídící desku}, pro první učení je nejlepší \hyperref[alks]{ALKS}, pro vlastní stavbu robota potom vyberete desku podle požadavků na elektroniku  
\item výhledově cokoli dalšího, co chcete připojit na robota (serva, ultrazvuk, senzory a~motory všeho druhu ... )
\end{itemize}


\vspace*{1ex}

\noindent {\bf Znalosti:  }
\begin{itemize} 
\item  běžná práce se soubory ve vašem operačním systému (hledání, kopírování, mazání, ... )
\item  velmi se hodí schopnost porozumět textu psanému v~jednoduché angličtině
\end{itemize}

\vspace*{1ex}

\noindent {\bf Taky se hodí vědět, že:  } 
\begin{itemize} 
\item veškeré \color{mygreen} zelené\color{black} \ a~\color{blue} modré \color{black} 
 \color{black} odkazy a~slova jsou proklikávací (s~výjimkou barevného zvýraznění 
syntaxe\footnote{syntaxe -- způsob zápisu daného jazyka, barevný zápis je mnohem přehlednější} ve zdrojových textech jazyka C++)
\item na začátku textu je (klikací) \hyperref[obsah]{obsah} 
\item na konci textu je   \hyperref[rejstrik]{rejstřík} %(také \fxnote*[author=JP]{TODO: zatím klikací není}{klikací})
\item v žádném textu o elektronice a robotice nemůže být všechno, pokud zde něco nenajdete, použijte např. Google; také může pomoci další   \hyperref[literatura]{literatura}
%\item tento text slouží pro \uv{začátečnický rozjezd}, pravděpodobně vám proto časem přestane stačit
\end{itemize}


\section{Přehled soutěží} \label{prehled_soutezi}  \index{Přehled soutěží} 

Plán účastnit se se svým robotem soutěže je nejen motivací pro začátek stavění robota, ale hlavně motivací pro jeho dokončení, protože soutěž nepočká.  
V současné době se soustředíme na tyto soutěžní dny: \href{http://www.robotiada.cz/}{Robotiáda}  v~Brně a \href{http://robotickyden.cz}{Robotický den}  v Praze. 
Dále existuje \href{http://www.robotika.sk/contest/2018/index.php}{Istrobot} v Bratislavě, kam téměř nejezdíme. 
Hlavní důvod je, že soutěž bývá v dubnu, kdy roboti ještě nejsou hotoví. 

U všech soutěží, kterých se rozhodnete zúčastnit, je potřeba důkladně nastudovat pravidla. Hlavní členové týmu by je měli znát víceméně zpaměti. 

Na \textbf{Robotiádě} \index{Robotiáda} je podmínka, že hlavní části robota (řídící systém, pohony, senzory) musí být z \hyperref[lego:stavba]{Lega}.  

 Robotiáda probíhá obvykle na začátku února a obsahuje několik soutěží. Každá z nich je rozdělená na kategorii ZŠ a SŠ. 
Které soutěže se zúčastníte, je v podstatě na vás, s vyjímkou Freestyle, kterou příliš nedoporučuji kvůli velmi nejistému výsledku. Pokud začnete v září, je potřeba se přípravě, stavbě a programování robota věnovat alespoň jedno odpoledne týdně. 

\textbf{Robotický den} \index{Robotický den} probíhá obvykle na začátku června a obsahuje také několik soutěží, které jsou rozdělené na kategorie \uv{Roboti pouze z dané stavebnice (např. z Lega)} a \uv{Roboti z čehokoliv}. 
Není zde dělení podle věku, takže ZŠ a SŠ soutěží v jedné skupině s VŠ a dospělými podle toho, jak se kdo přihlásí. 
Výběr vhodné soutěže je pro úspěch zásadní a je nezbytné jej konzultovat s~vedoucím kroužku.  
Pro stavbu funkčního robota je potřeba věnovat přípravě, studiu a stavbě minimálně jedno odpoledne týdně, pokud jste pomocný člen týmu a dvě odpoledne  týdně, pokud jste např. hlavní mechanik nebo hlavní programátor týmu. Čím víc, tím líp, protože \hyperref[denik]{zkušenosti} ukazují jasně, že času není nikdy dost. 

\subsubsection*{Přehled soutěží Robotického dne:}
\vspace{-0.7cm} 
\begin{description}
	\item[]
	\item[Doporučené soutěže:] Toy Cleanup Beginner (pokud se Robotického dne účastníte poprvé), Ketchup House, Bear Rescue Advanced, Line Follower, 
	\item[Težko říct:] Puck Collect, Roadside Assistance, Toy Cleanup Advanced
	\item[NEdoporučené soutěže:] RoboCarts, Free Style, Mini Sumo
\end{description}

Tato (ne)doporučení vycházejí ze zkušeností z předchozích let a odhadu reálných možností středoškolských studentů. 
 
\section{Stavba robota z Lego Mindstorms -- doporučení} \label{lego:stavba}

Na Robotický den můžete stavět pouze z \uv{čistého} Lega, viz  \href{http://robotickyden.cz/2018/rules/2018-Construction_Kits-parts.php}{pravidla}. 
Na Robotiádě můžete využít do konstrukce například díly z překližky  pro větší tuhost a snazší montáž. 
Překližkové díly se dají navrhnout například v modelovacím systému \href{https://www.onshape.com/}{Onshape}  a následně je možné je vyřezat na laseru na pracovišti \href{https://www.fablabbrno.cz}{Fablab}.

Pro programování robota z Lega máte dvě možnosti: 

\begin{description}
	\item[obrázkový kód]~\\
		použijete nativní obrázkové programování, ke stažení \href{https://www.lego.com/en-us/mindstorms/downloads}{zde};
		v podstavě jde o událostmi řízené programování -- viz kapitola \ref{prog:udalosti}
		%todo Jarkova videa? 
	\item[kód v C++]~\\
		použijete prostředí, které vám umožní programovat robota v C++, například \href{http://www.cpp4robots.cz/}{Cpp4Robots}
		-- viz kapitola \ref{Cpp4Robots}, potom %todo --- stavový automat 

		Pozor, toto prostřední pracuje pouze pod Windows7 a novějšími.
		
\end{description}



