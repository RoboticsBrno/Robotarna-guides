 Příkazy TeXu, které se můžou hodit:
\coltext\Black\Red{text ...} - obarví text i podklad

\nl - nový řádek 
\hfil \break - nový řádek
\vfil \break - nová stránka
\link[ref:PWM]{\Magenta}{PWM}\Black  - odkaz na text (je vidět fialově)
\dest[ref:PWM] - cíl odkazu (není vidět)
\ii<mezera>slovo<mezera> převezme slovo do rejstříku (slovo v textu není vidět) 
\iid <mezera>slovo<mezera> převezme slovo do rejstříku (slovo v textu je vidět)
viz TPP str. 62

\label[neco] \ref[neco] vždy před nadpisem sekce a spol 
Tady je normální text\rfc{ověřit, zda netvrdím blbosti}. 
viz http://petr.olsak.net/opmac-tricks.html#rfc  

\emergencystretch=2cm % když se text se nechce vejít do řádku a zobrazí černý obdélník, je potřeba zvětšit pružnost mezer

Příkazy LaTeXu, které se můžou hodit:


\fxnote*[author=JP]{Testovací poznámka k~prvnímu slovu ve větě - autor Jarek Páral (JP)}{Tento}

 \hyperref[label]{text}
 
 \href{http://www.kutac.cz/}{zde}
 
\cite[strana~49]{ard}
\cite[strana~49]{hr} 



-----------------------
na později:

\parskip=\medskipamount % mezi odstavci bude mezera jako \medskip
\parindent=0pt % odstavce nebudou odsazeny zará3kou
\let\itemskip=\relax % 3ádné dal1í mezery mezi výety


% test grafiky a přemazání nevhodného čísla v jednom - premazavani nefunguje, číslo je vidět i přes barvu  
\pdfliteral{q % uchování grafického stavu
0 1 0 RG 0 0 1 rg % nastavení barvy pro eáry (zelená)
20 w % (Width) 1íoka eáry 
200 -30 m % (Moveto) pero polo3íme do poeátku
250 -30 l % (Lineto) poidáme úseeku z 0 0 do 30 30
S % kresba samotné čáry 
Q % návrat do původního stavu 
}

