
\section{Onshape} \label{onshape}

\subsection{Úvod, přihlášení, nový dokument}

\textbf{Onshape} \index{Onshape}
 [onšejp] je relativně jednoduchý CAD program pro navrhování 3D modelů. Jeho ovládání je podobné programu SolidWorks. 
 
 \subsubsection{Možnosti použití}
 
 K čemu je pro nás Onshape dobrý? Můžeme v něm vyrobit: 
 
 \begin{description}
 	\item[--] hrubý návrh robota bez uvedení rozměrů pro debatu o konstrukci: \uv{bude to vypadat asi takhle a dělat asi tohle}
 	\item[--] podrobný návrh včetně všech rozměrů a výkresů pro vypálení dílů na laseru  
 	\item[--] cokoliv mezi tím
 \end{description}
 
 
Onshape je pro vzdělávací účely zdarma s tím, že všechno, co si v něm vytvoříte, je veřejně dostupné. 
Je dostupný přes webový prohlížeč (Opera, Firefox, Chrome) a proto funguje na všech operačních systémech. 
Podmínka je, aby prohlížeč měl zprovozněné WebGL rozhraní, což některé staré grafické karty nezvládají. 

Onshape je pouze anglicky, překlad některých pojmů je dále v textu.

Pokud s Onshape začínáte, je nutné si vytvořit účet. 
Na webu \url{onshape.com} klikněte na {\tt Sign in} (přihlášení) a po otevření přihlašovacího okna na {\tt Sign up} (založení nového účtu).


\subsubsection{Nový dokument}


\textbf{Dokument} je v Onshape obálka pro všechny soubory, které se týkají daného projektu. 

 \textbf{Projekt} je pro nás například konstrukce nového robota -- 3D model robota složený z jednotlivých dílů, vazby mezi těmito díly a výkresy všech dílů. 
 
 Po prvním přihlášení do Onshape klepněte vlevo nahoře na {\tt Create} a zvolte {\tt Document...} 

Zadejte název nového dokumentu (použijte pouze anglická písmena a číslice!) a potvrďte {\tt OK}.  

Pozn.: Stejně tak při pojmenovávání čehokoliv dalšího \textbf{používejte pouze anglická písmena a číslice}.

\subsubsection{Popis pracovního prostředí}

Otevře se hlavní okno programu. 
Nahoře jsou ikony pro úpravy dílů. 
Vlevo je panel se seznamem všech geometrických prvků v projektu (díly, skicy, pomocné roviny atd.). 
Uprostřed jsou tři hlavní roviny a počátek souřadnic (origin). 
Vpravo spíše nahoře je \uv{kostka}, která ukazuje, jak je vytvářený díl nebo sestava právě otočená. 

\subsubsection{Nastavení pracovního prostředí}

Nastavte si v Onshape stejné ovládání, jako je v SolidWorks: 
klikněte vpravo nahoře na svoje jméno, zvolte \texttt{My account}, vlevo \texttt{Preferences} a níže na stránce \texttt{View manipulation}.  


\subsubsection{Návody, nápověda}

Vpravo nahoře je tlačítko {\tt Learning Center}. Obsahuje velké množství krátkých videí, které vás programem krok za krokem provedou. 
\href{https://learn.onshape.com/courses/fundamentals-navigating-onshape}{Videa} jsou pouze anglicky, ale dobře srozumitelná. Pod každým videem je napsané všechno, co je ve videu řečeno. Pokud jenom trochu umíte anglicky, doporučuji je shlédnout a nebo přečíst, dá vám to hodně. 

Kompletní přehledná nápověda k Onshape je  \href{https://cad.onshape.com/help/Content/gettingstarted.htm?tocpath=Desktop\%20Help\%7C_____3}{zde}.

Pro všechny ikony v Onshape platí, že když na ně najedete myší, objeví se jejich název. 
Když počkáte několik sekund, objeví se nápověda. 


%\subsubsection{Cizí zdroje}

%Public - veřejné dokumenty 

%Když chci použít a upravit veřejný projekt někoho jiného, dám Copy dokument

%Dále viz kapitola \ref{sestava:vyroba}

\subsubsection{Označení}

Označení provedete kliknutím myši a tažením. Vytvoří se obdélník. Když táhnete myší zleva doprava, označí se pouze to, co je zcela uvnitř obdélníka. 
Když táhnete zprava doleva, označí se vše, co je alespoň částečně uvnitř obdélníka. \textbf{Od}značení všech označených dílů zajistí mezerník (klávesa {\tt Space}). 


\subsubsection{Posunutí, otočení a přiblížení}

Posunutí prvku: {\tt Shift}  + šipka směru, kam chceme posouvat nebo {\tt Ctrl} + stisknuté kolečko myši.  

Prvky je možné přiblížit nebo oddálit otáčením kolečka myši. Přitom se přibližujeme k bodu, na který právě myš ukazuje. 

Stisknutím kolečka a posunem myši se otáčí dané prvky. 

Chování popsané zde odpovídá nastavení \texttt{SolidWorks} (to doporučujeme, protože SolidWorks budete na SPŠ Sokolská časem probírat).


\subsubsection{Plocha, objem, hmotnost}

V části díly (Parts) vlevo dole označíte díly. Vpravo dole se objeví ikona \uv{váhy}. 
Klepnutím na ni se otevře okno, kde je spočtená plocha povrchu (Surface area) a objem (Volume) označených dílů. 
Pokud je zadaný materiál (klikněte pravým tlačítkem na díl v seznamu vlevo dole, z menu vyberte {\tt Assign material...}), zobrazí se i hmotnost (mass) a další parametry.


\subsection{Postup práce v Onshape}
%Postup tvorby modelu / Postup vytváření podkladů pro výrobu robota
 
\begin{enumerate}
	\item pro každý díl vytvoříte skicu ve 2D -- kapitola \ref{skica:vyroba}
	
	\item ze skicy vytvoříte díl ve 3D -- kapitola \ref{dil:vyroba}
	
	\item díly poskládáte do sestavy a zkontrolujete, že k sobě správně pasují -- kapitola \ref{sestava:vyroba}
	
	\item vytvoříte DWG soubor ze všech dílů $\rightarrow$ podklad pro řezání na laseru -- kapitola \ref{laser:vykresy}	
\end{enumerate}



\subsection{Výroba skici} \label{skica:vyroba}


Skica (Sketch) \index{skica} je dvourozměrný podklad pro tvorbu dílů ve 3D. 

Klepnutím zvolte rovinu, ve které chcete skicu vytvářet. Klepnutím na tlačítko {\tt Sketch} vlevo nahoře založte ve zvolené rovině novou skicu. 
Zároveň se ikony nahoře změní z ikon pro úpravy dílů na ikony pro úpravy skici. 
Zvolte z nich např. kružnici, klepněte na počátek a tažením vytvořte kružnici na skice. 
Podobně můžete vytvořit úsečku nebo obdélník. 
Pomocí dalších ikon lze vytvořit mnohem složitější tvary -- více v kapitole \ref{skica:možnosti}.
Nejvíc se ale naučíte, když si všechny ikony vyzkoušíte. 

Pomocí \uv{kostky} vpravo nahoře zvolte vhodnou orientaci skici. 

Abychom mohli ze skici nebo modelu vyrobit výkres, musí být tzv. úplně určená.

Skica je \textbf{úplně určená}, když má zadané všechny rozměry a také polohu (vzdálenost) od počátku nebo od bodu nebo čáry, která je vztažená k počátku.
Úplně určená skica je černá, dokud není úplně určená, má neurčené čáry modré.  

Rozměry zadáváte pomocí ikony \uv{kóta}.
Při zadávání rozměrů a polohy se velmi doporučuje využívat vazeb a proměnných -- více viz  \ref{skica:možnosti}.


Rozměry se vloží tak, že se číslo napíše ihned po dokončení daného geom. prvku do skici 
nebo se může doplnit později po kliknutí na ikonu kóty (dimensions).

Rozměry se průběžně zobrazují vpravo dole po klepnutí na prvek (úsečka, plocha), jehož rozměr nás zajímá. 

Hotovou nebo rozpracovanou skicu uzavřeme pomocí zeleného zatržítka. 

Mimochodem, Onshape nemá Save, vše je automaticky ukládáno do cloudu.

Při větším počtu skic a dílů doporučuji skici a díly smysluplně přejmenovávat (opět pouze anglická abeceda a čísla): klikněte pravým tlačítkem na název dílu a zvolte {\tt Rename}.

\subsection{Možnosti skici} \label{skica:možnosti}


\subsubsection{Proměnné}

Onshape umí přiřadit název danému číslu -- vytvořit \textbf{proměnné} (variable). \index{proměnné}
To znamená, že na začátku práce si dané číslo (např. tloušťka překližky) nazvete v menu pro díly pomocí ikony {\tt Variable}, např. {\tt tloustka  3 mm}. 
Tento název potom používáte všude v dokumentu. 
Název musí být zavedený před jeho prvním použitím (musí být v menu vlevo výše, než všechny skici nebo  díly, které ho používají).
Když se posléze ukáže, že potřebujete překližku o síle 4 mm, stačí tuto hodnotu změnit na jediném místě. 
To je obrovská pomoc, pokud například při konstrukci používáte tzv. zámečky -- a to byste měli, pokud má robot držet pohromadě a být pevný. 
%příklad na zámečky mám v dokumentu Test2 

\subsubsection{Vazby}

Vazby ve skice můžete zobrazit nebo skrýt pomocí zatržítka {\tt Show constraints} -- třetí řádek pod zeleným zatržítkem, které uzavírá skicu. 

Pro vložení vazby (constrain) do skici jsou určeny ikony vpravo od ikony kóty. 
Přehled důležitých vazeb (ikony zleva doprava): 

\textbf{shodnost} (Coincident)  = dvě stejné entity (body, kružnice, ... ) se sloučí do jedné

\textbf{středová souměrnost} (dvou kružnic)  Concentric 

\textbf{rovnoběžnost} (Parallel)

\textbf{tečna} (Tangent)

\textbf{vodorovný směr} (Horizontal)
 
\textbf{svislý směr} (Vertical)

\textbf{kolmost} (Perpendicular) 

\textbf{stejný rozměr} (Equal)

\textbf{střed úsečky} (Midpoint)

Další ikony se už moc nepoužijí s vyjímkou \textbf{osové souměrnosti} (Symetric), kde se nejdřív vybírá osa, potom čáry, které se mají zrcadlit.

Například když víte, že budete mít v podvozku 8 stejných děr pro uchycení sloupků, zadáte rozměr pouze první z nich a ostatním zadáte vazbu stejný rozměr.
Když je pak nutné změnit průměr sloupku, provedete změnu pouze na jednom místě. 

\subsubsection{Další možnosti}

Některé další možnosti při úpravě skici: 

trim -- vystřihnutí dané křivky "od bodu k bodu"

fillet -- zaoblení 

offset -- zdvojení hran a jejich odsazení 

mirror -- zrcadlení = osová souměrnost -- nejdřív se vybírá osa, potom čáry, které se mají zrcadlit

linear pattern -- dvourozměrné lineární pole, 
pod ním je ještě kruhové pole a otočení/transformace 
 

\subsection{Výroba dílu  -- poznámky} \label{dil:vyroba} \index{díl}

Nový díl založíte pomocí tlačítka \texttt{+} vlevo dole. 
Z menu vyberete {\tt Create Part Studio}. 
Otevře se nová záložka, ve které pomocí skici začnete tvořit nový díl.  

Ze skici vytvoříte díl (part) ve 3D, nejčastěji pomocí příkazu vytažení (Extrude) -- první ikona zleva na panelu ikon. 

U skic lze při výrobě dílu průběžně zapnout a vypnout viditelnost (ikona \uv{očičko} vpravo u názvu dílu).

Editaci dílu uložíme klepnutím na zelené tlačítko podobně jako u skici.

Barva dílu: klikněte na díl (seznam vlevo) pravým tlačítkem a  z menu vyberte {\tt Edit appearance}. 

V jedné záložce \uv{Part studio} můžete vytvořit více dílů -- to se doporučuje, pokud budou spolu díly úzce souviset a rozměry jednoho dílu využijete při návrhu druhého dílu. 

%potisk dílu (gravírování) - ve skice volba text


\subsection{Výroba sestavy} \label{sestava:vyroba} \index{sestava}

Novou sestavu (assembly) založíte pomocí tlačítka \texttt{+} vlevo dole. Z menu vyberete {\tt Create Assembly}.

Každá sestava může být \textbf{podsestavou} (částí) jiné sestavy. \index{podsestava}

Díly, skici, povrchy nebo podsestavy se vkládají do sestavy pomocí tlačítka insert vlevo nahoře. 
První díl v sestavě se musí ručně upevnit (fix) vůči počátku sestavy. 
Umístěte díl podle potřeby, potom klikněte pravým tlačítkem na díl a zvolte {\tt fix}.

Do sestavy můžete vkládat díly, skici, povrchy nebo sestavy, svoje i kohokoliv jiného. 

Pokud se stejný díl vkládá do jedné sestavy vícekrát (šroubky, kola, atd. ), nazývají se jednotlivé vložené části \textbf{instance} dílu. 
Pokud změníte díl, změní se všechny instance dílu v sestavě.
Další instance stejného dílu se vloží takto: vlevo v soupisu dílů klikněte pravým tlačítkem na díl a zvolte {\tt Copy}. 
Potom klikněte opět pravým tl. do plochy, kde tvoříte sestavu a zvolte {\tt Paste}. Můžete taky použít klasické {\tt Ctrl+C, Ctrl+V}. 

Pohled na díly v řezu (section view): klikněte na \uv{malou kostku vpravo} a vyberte {\tt Turn section view on}, následně vyberte rovinu, podle které má řez probíhat. 

\subsubsection{Vložení dílů ze SolidWorks}

Do sestavy jde vložit i díly vymodelované v SolidWorks (přípona .SLDPRT). 
Tyto se napřed musí importovat (klik na {\tt Onshape} vlevo nahoře, potom na {\tt Create } pod ním a zvolit {\tt Import files...}) 
Importovaný díl vytvoří vlastní dokument. Potom se musí u tohoto dokumentu vytvořit alespoň jedna verze (poklepání otevřete dokument, vlevo nahoře mezi {\tt Onshape} 
a názvem projektu jsou tři ikony, klikněte na prostřední a použijte tlačítko {\tt Create version} ). 

\subsubsection{Vazby v sestavě}


Díly na skutečném robotovi jsou spolu spojeny, nejčastěji napevno nebo se mohou navzájem otáčet.  
V Onshape se takové upevnění zadává pomocí tzv. \textbf{vazby} (mate).
Jde o jiné vazby, než ve skice a angličtina pro ně má jiný název.  
Pokud byste vazby nezadali, budou díly v sestavě \uv{plavat} nezávisle jeden na druhém. 

Přehled možných vazeb mezi díly a jejich použití je \href{https://cad.onshape.com/help/Content/mate.htm?TocPath=Desktop%20Help|Assemblies|Mates|_____0}{zde}.
Využijete především pevné spojení (Fastened mate) a otáčení kolem osy (Revolute mate). 
Ikony pro všechny vazby najdete na horním panelu ikon (pokud jste přepnutí na záložku sestavy).


\subsection{Výroba výkresů/příprava pro laser} \label{laser:vykresy}


%todo přibude ... 


%todo výstup na 3D tisk -- formát stl 
%pravým tlačítkem na díl, Export, Format nastavit na STL, zjistit, jestli Text nebo binary a zjistit vhodné jednotky délky 


\subsection{Slovníček pro Onshape}

assembly -- sestava nebo podsestava, například celý robot nebo podvozek

constrain -- vazba v rámci skici

dimensions -- kóty = rozměry 

extrude -- vytažení 

fillet -- zaoblení 

linear pattern -- dvourozměrné lineární pole 

mate -- vazba v rámci sestavy (složené z dílů)

mirror -- zrcadlení = osová souměrnost - nejdřív se vybírá osa, potom čáry, které se mají zrcadlit

offset -- zdvojení hran a jejich odsazení 

origin -- počátek soustavy souřadnic

part -- díl = součástka

part studio -- tady se vytváří nové součástky 

sketch -- skica = nákres 

trim -- vystřihnutí dané křivky \uv{od bodu k bodu}










