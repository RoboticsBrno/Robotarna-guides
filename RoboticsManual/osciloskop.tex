

\section{Osciloskop}

\subsection{Než začnete}

%pokud někdo nastaví jako jazyk korejštinu a spol: tl Help, potom třetí tl. zleva pod obrazovkou vyvolá nabídku jazyků \rightarrow zvolím zpět angličtinu 

\textbf{Osciloskop} měří velmi rychle napětí. Dokáže si napětí pamatovat, zobrazit závislost napětí na čase a v zobrazeném průběhu napětí je možné změřit řadu parametrů, např. frekvenci. 


Máme možnost pracovat s digitálním paměťovým osciloskopem
\href{https://www.keysight.com/en/pdx-x201837-pn-DSOX2024A/oscilloscope-200-mhz-4-analog-channels?nid=-32542.1150190\&cc=CZ\&lc=eng\&pm=ov}{Agilent/Keysight DSO-X 2024A}.
Tento osciloskop má čtyři analogové vstupy (=kanály), takže lze současně měřit a zobrazovat čtyři signály. Každý kanál má svou barvu, se kterou se signál zobrazuje. 
Na každém kanálu může měřit napětí až do 300 V. %(RMS -- efektivní hodnota )

Přitom na vodorovné ose se zobrazuje čas, na svislé ose měřené napětí. 

Maximální zobrazované napětí se nastavuje pro každý kanál zvlášť, čas je pro všechny kanály vždy stejný.  

Osciloskop není stavěný pro přesné měření napětí, spíše orientační, protože na vstupu je pouze 8 bitový převodník napětí, ale měří přesně časy.


Dále je možnost měřit až 8 digitálních vstupů, přitom digitální vstupy umí pouze zobrazit a měřit časové parametry,  
dekódovat digitální sběrnici umí pouze analogové vstupy. 

Všechna ovládací kolečka na osciloskopu se dají také stisknout.

Když podržíte libovolné tlačítko nebo kolečko  2 sekundy, objeví se k němu podrobná nápověda (anglicky). 
Menu ke každému tlačítku se zobrazuje vždy dole na obrazovce a ovládá se tlačítky pod obrazovkou.  
Tlačítko Help zobrazí menu, které  obsahuje mimo jiné položky \texttt{Getting started} a  \texttt{Training singals}. 

\textbf{Vzorkovací frekvence} -- počet měření za sekundu. Lze nastavit až 2 GSa/s 
(G -- giga, Sa -- sample = jednotlivé měření ).

Vzorkovací frekvence se doporučuje nejméně 10x analogová šířka pásma, aby se ze singálu dalo něco poznat. 
Jinými slovy, pro měření signálu o frekvenci 50 kHz potřebuji nastavit vzorkovací frekvenci minimálně 500 kSa/s.

\subsection{Sondy}

\textbf{Sondy} (probe) jsou měřící kabely připojené k osciloskopu. 

Sondy  mají klobouček s háčkem pro snadné uchopení měřeného drátu. Když se klobouček sundá, lze měřit dotekově hrotem. 
Každá sonda má také pobočný drát zakončený \uv{krokodýlem}. 
Ten se připojuje vždy na zem měřeného obvodu. 
Pokud ho nepřipojíte, bude se vám na kabelu sondy indukovat šum z okolí, který často překryje vlastní signál. 

Červený křížek na sondách slouží ke zkalibrování sond pomocí vestavěného otočného kondenzátoru a signálu \texttt{Probe}, který generuje osciloskop. 
Kalibraci sond obvykle provádět nemusíte, stačí ji udělat při prvním použití sond. 

Sondy jsou obvykle nastavené tak, že dělí vstupní signál 10 (sonda x 10), v osciloskopu se pak nastavuje opětovné vynásobení, 
aby se zobrazovala správná hodnota. 

Osciloskop si pamatuje poslední nastavení, takže při zapnutí není nutné sondy znovu nastavovat. 


\subsubsection{Menu sondy}

\textbf{probe}: dělička v sondě (musí se nastavit stejně jak na sondě), mělo by to být, obvyklá hodnota je 10:1

zapnu střídavou vazbu: odstraní stejnosměrnou složku signálu %todo ????

\textbf{invert}: zobrazuje kladnou složku dolů místo nahoru 

\textbf{BW limit} potlačuje signály nad 20 MHz (tuto hodnotu na tomto osc. nelze měnít) $\rightarrow$ redukuje šumy, které nás obvykle nezajímají  


\subsection{Zahájení měření}

Tlačítko \texttt{Run/stop} -- červená neměří, zelená měří. Osciloskop ukládá naměřené hodnoty do naplnění paměti, potom nejstarší hodnoty zahazuje a přidává nejnovější. 
Tlačítko \texttt{Single} -- žlutá svítí -- osciloskop udělá právě jednu sadu měření, kterou naplní obsah paměti a dál neměří. 

Měření na sondě se zapíná tlačítkem s  číslem sondy (mezi velkým a malým otočným kolečkem).

Aby se vám měřený signál správně zobrazil, potřebujete mít optimálně nastavené rozlišení jak času, tak napětí. %todo doplnit nastavení obojího přesně
Pokud si nejste jistí  nastavením osciloskopu, stiskněte tlačítko \texttt{Auto Scale} a osciloskop se pokusí rozlišení nastavit sám. 



%\texttt{Time mode}: \texttt{normal}  -- obvyklé měření nebo %todo doplnit
%??? -- měření v reálném čase -- vhodné pro pomalá měření (točení potenciometrem )


Nastavení, od kdy přesně má osciloskop začít měřit, je v mnoha případech klíčové. 


\textbf{Trigger} -- říká: teď začni měřit. Trigger může mít pouze jeden vstup, který lze velmi různě navolit pomocí tlačítka \texttt{Trigger}.
 


\subsubsection{Nastavení času}
 
velké kolečko -- nastavení šířky periody, 
malé kolečko -- nastavení posunu vůči počátku 
tl. lupa -- hodně zvětší 

posun počátku času vlevo, střed, vpravo: tl. Horiz menu: Time Ref 

\subsubsection{Nastavení napětí}

velké kolečko -- zesílení 
malé -- posun nuly (= offset)
tlačítko Mode Coupling %todo doplnit
tlačítko Force Trigger -- okamžitě zahájí měření (v normal módu) 

Dva signály, které nejsou ze stejných hodin (stejného čipu), se obvykle na monitoru posunují vůči sobě. V takovém případě vypněte 
měření (tlačítko \texttt{Stop}) a změřte Off-line na obrazovce, co potřebujete.



\subsection{Význam některých tlačítek -- heslovitě}

Tlačítko \texttt{Default setup} uvede osciloskop do výchozího nastavení (doporučeno na začátek práce).

Lze uložit až 10 svých nastavení osciloskopu a podle potřeby se k nim vracet.  %todo  jak ? ******************

 Tlačítko  \texttt{Wave  gen} -- modře svítí -- zapnuto / nesvítí vypnuto. 
 Generátor funkcí je popsán v nápovědě osiloskopu (stiskněte \texttt{Wave  gen} a držte 2 sekundy). 

 Tlačítko \texttt{Meas}  -- menu měření.
Zde si nastavíte, co všechno chcete měřit  (až 4 veličiny zaráz).

Tlačítko \texttt{Cursors}   (=pravítka, dvě vodorovná a dvě svislá)  -- lze tím měřit zcela manuálně cokoliv na obrazovce.
Nejčastější použití -- sledování PWM a signálů na sběrnicích. 

Tlačítko \texttt{Refs}  (referenční signály) -- umí si pamatovat dva signály a srovnávat s nimi aktuální průběh  %todo ************************ jak se s tím pracuje 

Tlačítko \texttt{Math}  umí arit. výpočty se dvěma signály, taky umí Fourrierovu analýzu signálu. 

Tlačítko  \texttt{Digital} -- nastavení měření na digitálních vstupech.

Tlačítko   \texttt{Serial} -- serial decode mode  
%Tlačítko/menu ???? \texttt{} treshold --při dekódování sběrnic $\rightarrow$ nastavím, kde končí log. O a Log. 1  



%--------------

%Některé položky z menu:

%\begin{description}
%	\texttt{Mode}: \texttt{auto} -- čeká \uv{nějakou dobu} na signál z triggeru a pak začne měřit  
%	\texttt{normal} -- čeká na signál z triggeru neomezeně dlouho 
%	
%	
%	\item  \texttt{Coupling DC} -- měří všechno 
%	
%	\item  \texttt{Coupling AC} -- potlačí stejnosměrné složky signálu 
%	
%	\item \texttt{LF Reject} -- potlačení pomalých frekvencí 
%	
%	
%\end{description} 


%\texttt{HF Reject} -- potlačení vysokofrekvenční složky 

%\texttt{Noise Rej} -- potlačení šumu   %nepoznal jsem rozdíl  - pouze na "rozumně tvrdém" signálu 

%\texttt{Holdoff} -- nastavení doby, po kterou osciloskop ignoruje další signály %todo z triggeru nebo celkově? 

%\texttt{treshold} -- nastavení úrovně, jejíž překročení spouští měření






