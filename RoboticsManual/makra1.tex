% viz http://petr.olsak.net/opmac-tricks.html

% toto makro je potřeba pro text níže 

\newdimen\hyphdim
\let\hyphentt=\tentt  \hyphdim=\fontdimen6\hyphentt \divide\hyphdim by2

\def\hyphenprocess#1{\def\tmp{#1}\let\listwparts=\undefined 
   \setbox0=\vbox\bgroup\hyphenpenalty=-10000 \hsize=0pt \hfuzz=\maxdimen
      \edef\hychar{\hyphenchar\hyphentt=\the\hyphenchar\hyphentt}\hyphenchar\hyphentt=45
      \def~{\nobreak\hskip.5em\relax}\tt\noindent\hskip0pt\relax #1\par \hychar
      \hyphenprocessA
   \expandafter\let\expandafter\listwparts\expandafter\empty
      \expandafter\hyphenprocessB\listwparts
}
\def\hyphenprocessA{\setbox2=\lastbox 
   \ifvoid2 \egroup \else \unskip \unpenalty 
      \setbox2=\hbox{\unhbox2}%
      \tmpnum=\wd2 \advance\tmpnum by100 \divide\tmpnum by\hyphdim
      \ifx\listwparts\undefined \xdef\listwparts{,}%
      \else \advance\tmpnum by-1 \xdef\listwparts{\the\tmpnum,\listwparts}\fi
              \expandafter\hyphenprocessA \fi
}
\def\hyphenprocessB#1,{\if^#1^\expandafter\hyphenprocessC
   \else \tmpnum=#1 \expandafter\hyphenprocessD\tmp\end
   \fi
}
\def\hyphenprocessC{\expandafter\addto\expandafter\listwparts\expandafter{\tmp}}   
\def\hyphenprocessD#1#2\end{\addto\listwparts{#1}%
   \advance\tmpnum by-1
   \ifnum\tmpnum>0 \def\next{\hyphenprocessD#2\end}%
   \else
   \def\tmp{#2}\def\next{\addto\listwparts{\-}\expandafter\hyphenprocessB}\fi
   \next
}


%V následujícím triku ukážeme makro \coltext\BarvaA\BarvaB{text}, které vytvoří uvnitř odstavce text v bavě \BarvaA podbarvený barvou \BarvaB, přitom tento text podléhá řádkovému zlomu jako běžný text v odstavci. Například:

%Zde je běžný text odstavce.
%\coltext\Yellow\Blue{Tady je podbarvený textík, který se láme do řádků.}
%A pokračujeme v běžném textu odstavce.


\def\coltextstrut{height2ex depth.6ex}
\def\coltext#1#2#3{{\localcolor\let\Tcolor=#1\let\Bcolor=#2\relax
   \setbox1=\hbox{-\kern.05em}%
   \setbox1=\hbox{{\Bcolor\vrule\coltextstrut width\wd1}\kern-.05em\llap{\Tcolor -}}%
   \def\-{\discretionary{\copy1}{}{}}%
   \def\uline##1{\skip0=##1\advance\skip0 by.05em
      \Bcolor\leaders \vrule\coltextstrut\hskip\skip0 \hskip-.05em\relax}%
   \def\uspace{\fontdimen2\font plus\fontdimen3\font minus\fontdimen4\font}%
   \def~{\egroup\hbox{\uline{\wd0}\llap{\Tcolor\copy0}}\nobreak{\uline\uspace}\relax \setbox0=\hbox\bgroup}%
   \leavevmode\coltextA #3 {} }}
\def\coltextA#1 {\ifx^#1^\unskip\unskip\else
   \hyphenprocess{#1}\expandafter\coltextB\listwparts\-\end
\expandafter\coltextA\fi}
\def\coltextB#1\-#2\end{\ifx^#2^\coltextC{#1}\else
   \coltextD{#1}\def\next{\coltextB#2\end}\expandafter\next\fi}
\def\coltextC#1{\setbox0=\hbox{#1}\hbox{\uline{\wd0}\hbox{\llap{\Tcolor\copy0}}}\uline\uspace\relax}
\def\coltextD#1{\setbox0=\hbox{#1}\hbox{\uline{\wd0}\llap{\Tcolor\copy0}}\-}

%%%%%%%%%%%%%%%%%%%%%%%%%%%%%%%%%%%%%%%%%%%%%%%%%%%%%%%%%%%%%%%%%%%%%

%Zvýraznění syntaxe (syntax highlighting) nějakého programovacího jazyka mezi \begtt a \endtt provedeme pomocí \hisyntax{jazyk}, což napíšeme těsně před \begtt nebo před \verbinput. Toto je makro pouze pro jazyk C

\def\hisyntax#1{\bgroup\def\ptthook{\csname hisyntax#1\endcsname}}
\def\ghisyntax#1{\def\ptthook{\bgroup\csname hisyntax#1\endcsname}}

\def\hisyntaxC#1\egroup{\let\n=\relax \let\NLh=\relax \let\U=\relax
   \let\TABchar=\relax % pro OPmac trik 0151
   \adef{ }{\n\ \n}\adef\^^M{\n\NLh\n}\edef\tmpb{#1\egroup}%
   \replacestrings{\n\egroup}{\egroup}%
   \replacestrings{/*}{\tmp}\def\tmp##1*/{\U{commentC}{##1}}\edef\tmpb{\tmpb}%   
   \replacestrings{//}{\tmp}\def\tmp##1\NLh{\U{commentCpp}{##1}}\edef\tmpb{\tmpb}%
   \edef\tmp{\noexpand\replacestrings{\string\"}{\n{\string\"}}}\tmp
   \replacestrings{"}{\tmp}\def\tmp##1\tmp{\U{stringC}{##1}}\edef\tmpb{\tmpb}%
   \doreplace{##1}{\n{\chCcolor##1}\n}\charsC{}%
   \doreplace{\n##1\n}{\kwC{##1}}\keywordsC{}%
   \edef\tmp{\noexpand\replacestrings{\NLh\n\string##}{\NLh\noexpand\preprocC}}\tmp
   \doreplace{\n##1}{\numberC##1}0123456789{}%
   \let\NLh=\par \def\n{}\def\U##1{\csname##1\endcsname}%
   \tentt\localcolor\tmpb\egroup}
\def\doreplace#1#2{\def\do##1{\ifx^##1^\else \replacestrings{#1}{#2}\expandafter\do\fi}%
   \expandafter\do}
\def\printttline{\llap{\sevenrm\Black\the\ttline\kern.9em}} % \Black added 

\def\commentC#1{{\Green/*#1*/}}
\def\commentCpp#1{{\Green//#1}\NLh}
\def\stringC#1{{\def\ {\char32 }\Magenta"#1"}}
\def\preprocC#1\n{{\Blue\##1}}
\def\numberC#1\n{{\Cyan#1}}
\edef\keywordsC{{auto}{break}{case}{char}{continue}{default}{do}{double}%
   {else}{entry}{enum}{extern}{float}{for}{goto}{if}{int}{long}{register}%
   {return}{short}{sizeof}{static}{struct}{switch}{typedef}{union}
   {unsigned}{void}{while}} % zde jsou všechna klíčová slova jazyka C
\def\kwC#1{{\Red#1}}
\edef\charsC{()\string{\string}+-*/=[]<>,:;\percent\string&\string^|!?}
\let\chCcolor=\Blue

%%%%%%%%%%%%%%%%%%%%%%%%%%%%%%%%%%%%%%%%%%%%%%%%%%%%%%%%%%%

%Opmac nabízí jen třístupňovou hiearchii nadpisů: kapitoly, sekce a podsekce. Často se sektávám s požadavkem zařadit i možnost tisku podpodsekcí. Osobně to nepovažuji za příliš účelné a zbytečně to mate čtenáře. Ale je-li k tomu pádný důvod, můžete si vytvořit makro \seccc takto:

\newcount\secccnum

\eoldef\seccc#1{%
   \ifx\prevseccnum\theseccnum \global\advance\secccnum by1
   \else \global\let\prevseccnum=\theseccnum \global\secccnum=1
   \fi
   \edef\thesecccnum{\theseccnum.\the\secccnum}%
   \printseccc{#1}%
}
\def\printseccc#1{\norempenalty-100 \medskip
   {\bf \noindent \thesecccnum\quad #1\nbpar}%
   \nobreak \smallskip \firstnoindent
}

%^Makro řeší uvedený požadavek co nejjednodušším způsobem: zavádí číslování v registru \secccnum a tiskne ve fontu \bf (nezvětšená velikost). Neřeší dopravení informací do obsahu (tam je to stejně nevhodné) ani do záhlaví. Byl-li by takový požadavek, je potřeba využít makro \wcontents pro obsah a primitiv \mark pro záhlaví.

%%%%%%%%%%%%%%%%%%%%%%%%%%%%%%%%%%%%%%%%%%%%%%%%%%%%%%%%%%%%%%%%%%%%%%%%%%%%

%Důvod, proč OPmac neimplementuje aktivní čísla stránek v rejstříku jsou dva: drží se kréda ,,v jednoduchosti je síla`` a šetří pamětí TeXu. Za každý odkaz na stránku by totiž do makroprostoru musel přidat tři tokeny. Při tisíci heslech a průměrně třech odkazech u hesla dostáváme 9 tisíc tokenů navíc. Tomu se chce OPmac vyhnout.

%Pokud přesto chcete mít čísla stránek v rejstříku aktivní, stačí použít tento kód:

\def\printiipages#1&{\usepglinks#1\relax\par}
\def\usepglinks{\afterassignment\usepglinksA \tmpnum=}
\def\usepglinksA{\pglink{\the\tmpnum}\futurelet\next\usepglinksB}
\def\usepglinksB{\ifx\next\relax \def\next{}\else
   \ifx\next,\def\next,{, \afterassignment\usepglinksA \tmpnum=}\else
   \ifx\next-\def\next--{--\afterassignment\usepglinksA\tmpnum=}%
   \fi\fi\fi\next}
   
   
%%%%%%%%%%%%%%%%%%%%%%%%%%%%%%%%%%%%%%%%%%%%%%%%%%%%%%%%%%%%%%%%%%%%%%%

% Slova v restříku chceme mít oddělena podle jejich prvního písmene do oddílů, každý oddíl má být uvozen ,,nadpisem``, který obsahuje zvětšené písmeno, kterým všechna slova daného oddílu začínají.

\def\lastchar{}
\def\everyii{\expandafter\makecurrchar\currii\end
   \ifx\currchar\lastchar \else
      \bigskip{\typosize[10/12]\bf\currchar}\par\nobreak\medskip\noindent
      \let\lastchar=\currchar
   \fi
}
\def\makecurrchar#1#2\end{\uppercase{\def\currchar{#1}}}

%%%%%%%%%%%%%%%%%%%%%%%%%%%%%%%%%%%%%%%%%%%%%%%%%%%%%%%%%%%%%%%%%%%%%%%%%
% Interní poznámky viz http://petr.olsak.net/opmac-tricks.html#rfc
% Tady je normální text\rfc{ověřit, zda netvrdím blbosti}. \draft  
\newcount\rfcnum

\def\rfclist{}  
\def\rfcactive#1{\global\advance\rfcnum by1
   \dest[rfc:rfc-\the\rfcnum]\global\addto\rfclist{\rfcitem #1}}
\def\rfc#1{}
\def\rfcitem{\advance\rfcnum by1
   \medskip\noindent\llap{\link[rfc:rfc-\the\rfcnum]{\localcolor\Red}{[rfc-\the\rfcnum] }}}

\addto\draft{\let\rfc=\rfcactive}

\def\makerfc{\ifx\rfc\rfcactive
   \vfil\break {\secfont \noindent Requests for correction}\par
   \bgroup\rfcnum=0 \rfclist\egroup\fi}
   
%%%%%%%%%%%%%%%%%%%%%%%%%%%%%%%%%%%%%%%%%%%%%%%%%%%%%%%%%%%
% centuje poslení řádek odstavce 
\def\raggedcenter{%
  \parindent=0pt \rightskip0pt plus1em \leftskip0pt plus1em
  \spaceskip.3333em \xspaceskip.5em \parfillskip=0pt
  \hbadness=10000 % Last line will usually be underfull, so turn off
                  % badness reporting.
}
 
 %%%%%%%%%%%%%%%%%%%%%%%%%%%%%%%%%%%%%%%%%%%%%%%%%%%%%%%%%%%%%
% centrování textu, pokud mezi řádky vložím prázdný řádek, centruje se každý řádek zvlášť 
 \def\startcenter{%
  \par
  \begingroup
  \leftskip=0pt plus 1fil
  \rightskip=\leftskip
  \parindent=0pt
  \parfillskip=0pt
}
\def\stopcenter{%
  \par
  \endgroup
}
\long\def\centerpars#1{\startcenter#1\stopcenter} 

%You can now choose between

%\startcenter
%Some text to be centered
%\stopcenter
%and

%\centerpars{Some text to be centered}
%An empty line in the text will end the line.
 
