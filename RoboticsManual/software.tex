 
 \section{Lorris} \label{lorris}
 
 {\bf Lorris}\footnote{\url{http://tasssadar.github.io/Lorris/cz/index.html}} \index{Lorris} je rozsáhlá sada nástrojů, které mají společný cíl -- pomáhat při vývoji, ladění a řízení zejména robotů, ale i jiných elektronických zařízení. V současnosti neexistuje jiná volně dostupná aplikace, která by umožňovala dostatečně jednoduše v téměř libovolném formátu zobrazit data přicházející z čipů nebo i data ze souborů. 
 
\paragraph{ Hlavní možnosti použití: }

\begin{itemize}
	\item grafické zobrazování, přiřazování a analýza (binárních) dat z čipů
	\item vykreslování příchozích dat v grafu 
	\item zpracování dat z jednoho zdroje ve více modulech současně
	\item zobrazení dat z více zdrojů na jednom místě -- ideální pro ladění komunikace mezi více zařízeními 
	\item možnost libovolných úprav příchozích i odchozích dat pomocí skriptů v jazyce Python 
	\item simulace chovaní plánovaného robota -- hledání a ladění strategií
	\item simulace dat ze senzorů (zatím) neexistujícího robota pro psaní a ladění programu 
	\item vytváření vlastních ovládacích prvků -- například ovládání robota joystickem z počítače 
	\end{itemize}

 Lorris naprogramoval Vojtěch Boček a popsal ji podrobně ve své práci SOČ:
  \url{http://soc.nidv.cz/archiv/rocnik35/obor/18}.
  
 Video s krátkým představením Lorris: \url{http://www.youtube.com/watch?v=LkmFn40BbX8}.
 
  \hyperref[prog:lorris]{Příklad} posílání dat pro Lorris.
 
\section{\LaTeX} 

\subsection{Proč používat \LaTeX} 

Tento text je psán v~sázecím systému \index{LaTeX} \LaTeX . 
Jeho silná stránka je především matematická sazba 
(bohužel nevyužijeme) a~snadné zpracování obsahu, rejstříků, seznamů obrázků a~tabulek a~podobně, což dramaticky urychluje přípravu dokumentu. 

Čas, který vložíte do nastavení a~učení se systému, se vrátí v~rychlosti práce
 $\rightarrow$ jedná se o řešení vhodné pro delší dokumenty, např. pro soutěž SOČ nebo dlouhodobou maturitní práci\footnote{\url{https://github.com/RoboticsBrno/Latex-slideshow-czech}.}. 
 
Návody pro \LaTeX{}  lze najít na internetu,  pro úvodní zorientování 
 doporučuji text  \bfseries \itshape  \LaTeX{} pro pragmatiky\footnote{\url{http://mirrors.nic.cz/tex-archive/info/czech/latex-pro-pragmatiky/latex-pro-pragmatiky.pdf} }\upshape \mdseries .

Hodně vám také může pomoct zdrojový text \href{https://github.com/RoboticsBrno/RoboticsBrno-guides/tree/RoboticsManual/RoboticsManual}{této dokumentace}, 
 především
 \href{https://github.com/RoboticsBrno/RoboticsBrno-guides/blob/RoboticsManual/RoboticsManual/_dokumentace.tex}{hlavní soubor},
 kde je nastavení podrobně komentované.  

Příkazy \LaTeX{}u podobně jako C++ a~systémy typu Linux {\bf rozlišují velká a~malá písmena}. 
 

\subsection{Instalace a editory pro \LaTeX}

%\paragraph{Instalace}

Podobně jako Linux, je i \LaTeX dostupný v řadě distribucí. Doporučuji buď 
{\bf TeXLive}\footnote{\url{https://www.tug.org/texlive}} \index{TexLive}
 nebo {\bf MiKTeX}\footnote{\url{https://miktex.org/}}. \index{MiKTeX} 
{\bf }
%\paragraph{Editory}

Jako editory ve WinXP používám {\bf PSpad}\footnote{\url{http://www.pspad.com/}},\index{PsPad} v~linuxu {\bf TeXstudio}\footnote{\url{https://www.texstudio.org/}}. \index{TeXstudio} 
Oba editory umí zavolat překlad do pdf pomocí klávesové zkratky, zobrazit výsledný pdf a~barevné zvýraznění syntaxe. TeXstudio má navíc velké možnosti pro zrychlení práce.  


Další možností je tvořit latexové dokumenty online bez nutnosti instalace, například pomocí služby 
{\bf overleaf}.\footnote{\url{https://www.overleaf.com/}} Overleaf je online služba, která vám umožňuje psát, sdílet a komentovat LaTeXové dokumenty -- ideální pro psaní SOČ. Úvod do možností služby je 
\href{http://www.kutac.cz/blog/pocitace-a-internety/overleaf-online-latex-editor/}{zde}.  


\subsection{\LaTeX{} a SOČ}

Jarek Páral vytvořil na Overleafu LaTeXovou šablonu\footnote{Tady je k dispozici šablona, kterou si můžete na Overleafu zkopírovat jako vlastní projekt a začít tvořit: \url{https://www.overleaf.com/read/gvqvqzwgdtwk} }
 pro SOČ, kde je řada prvků sazby už optimálně přednastavená. 
 Šablona navíc obsahuje informace o tom, jak a co psát, o citacích a dalších věcech -- vřele doporučuji. 

Pro inspiraci také doporučuji práce Vojty Bočka, Honzy Mrázka, Jarka Párala, Bédi Saida a Martina Sýkory, dostupné v archivu SOČ\footnote{\url{http://www.soc.cz/archiv-minulych-rocniku/}}, ročníky 32. - 37., kategorie informatika a elektro.

\subsection{Další inspirace k SOČ}

	 \paragraph{Jak na psaní SOČ:}
		\begin{itemize}
			\item \url{http://www.soc.cz/soc-krok-za-krokem/}
			\item \url{http://www.jcmm.cz/cz/podpora-soc.html}		
		\end{itemize}

	\paragraph{Šablony SOČ pro Word}
	\begin{itemize}
		\item \url{http://www.soc.cz/dokumenty/sablona_SOC.docx}
		\item \url{http://www.jcmm.cz/data/sablona_pro_sockare.docx}
	\end{itemize}

Doporučuji si obě šablony přečíst, i když je třeba nepoužijete -- jsou v nich zajímavé a užitečné rady.


 
\section{Další software}



\subsection{Proficad}

Proficad\footnote{Instalační soubor seženete v~kroužku nebo na webu.} 
je software určený původně pro snadné a~rychlé kreslení elektronických schémat a~v~této oblasti je vynikající. 
Lze jej použít i~jako jednoduchý vektorový editor obrázků. \index{Proficad}

SPŠ Sokolská zakoupila plnou multilicenci pro Proficad, takže studenti i~učitelé jej mohou používat bez omezení. 

Ovládání programu je velice intuitivní a~nápovědu prakticky nepotřebujete -- s~jedinou výjimkou, a~tou je nastavení rastru. 
Po instalaci je rastr zobrazení automaticky nastaven na 2 mm. To znamená, že součástky 
můžete umisťovat například 10 mm nebo 12 mm od kraje, ale nic mezi tím. 
Většinou se to hodí -- součástky máte na schématu pěkně zarovnané -- ale 
někdy je prostě potřeba rastr například vypnout neboli nastavit na nulu. 
Nastavení rastru je schované zde:  {\it soubor/nastavení/dokument/obsah/rastr}.

\subsection{Python}

Programovací jazyk Python použijete, když chcete například napsat program pro počítač, který bude komunikovat s robotem. 
Nepoužívá se pro programování čipů v robotovi. 
Návody pro Python jsou například \href{https://www.sallyx.org/sally/python/}{zde} a  \href{http://diveintopython3.py.cz/index.html}{zde}. 


\subsection{Linux} 

Tato kapitolka není úvodem do Linuxu (materiálů na toto téma je na webu spousta).  Jsou zde poznámky, které hodí, když uživatel přechází z Windows na Linux. \index{Linux}

\paragraph{Některé rozdíly Linux -- Windows}

\begin{itemize}

\item Linux {\bf rozlišuje velká a~malá písmena}. V terminálu, v názvech souborů, všude. 

\item Když si chce uživatel nainstalovat Linux, vybírá ne operační systém, ale tzv. distribuci = operační systém + spousta předinstalovaných programů. Kritérií pro výběr je spousta, pár tipů, které používáme na Robotárně: 

	\begin{itemize}
		\item distribuce {\it Mint} má podobné rozložení grafických ovládacích prvků jak windows a je nenáročná na hardware 
	
		\item distribuce {\it Ubuntu} je standardní volba s velkou výbavou programů. Její odlehčená verze pro starší počítače je {\it Lubuntu}. 
	
	\end{itemize}  

\item V Linuxu se celkově mnohem víc (a účiněji) používá příkazová řádka neboli program  {\tt terminál}. \index{terminál} Spouští se klávesovou zkratkou {\tt Ctrl+Alt+T} nebo v Lubuntu spustíte správce souborů (2. ikona dole vlevo), přepnete se do požadovaného adresáře a stisknete {\tt F4}. \label{terminal}

\item Adresářová struktura Linuxu je jednotná pro všechny disky v počítači. Akce typu: {\it \uv{přepni se na disk e:} } se v terminálu provede jako {\it \uv{přepni se do adresáře /media/<uzivatel>/ ...  } }. Nebo se přepnete pomocí jiného programu, např. Správce souborů. 	
	
\item Některé programy fungují pod Linux i Windows, u jiných se při přechodu z Windows na linux musí na webu najít odpovídající náhrada, opět pár tipů: 

	\begin{itemize}
		\item Ve WinXP se pro prohlížení pdf souborů osvědčil Foxit Reader verze 2.3, v~linuxu Evince.  
			
		\item Irfan view jede i~v~Linuxu\footnote{\url{http://www.boekhoff.info/install-irfan-view-on-linux/}}.
	
	\end{itemize}

	
\end{itemize}


\paragraph{Co se hodí vědět }

\begin{itemize}
	
	
\item Heslo v Linuxu se při zadávání v terminálu nevypisuje ani tečkami (ale počítač o něm ví!). 

\item Znak zavináč se napíše {\tt AltGr+v}, znak vlna se napíše {\tt AltGr+a}.

\item Anglické znaky na české klávesnici napíšete, když použijete {\tt AltGr}. 

\item Drivery do tiskáren mají příponu ppd. Před jejich použitím se musí nainstalovat program cups. 

\item Snímek obrazovky uložíte stiskem klávesy {\tt PrintScreen} do adresáře 
{\tt /home/<prihlaseny uzivatel>}. 

\item Znak {\tt \#} znamená řádek na terminálu (zastupuje vypsání aktuálního adresáře). 

\item Zkratka {\tt Ctrl+C} v terminálu není pro kopírování, ale ukončení běžícího programu. (Mimo terminál kopírování funguje jako ve windows). Pokud chceme v terminálu kopírovat a vkládat, použijeme {\tt Ctrl+Shift+c}  {\tt Ctrl+Shift+v}.

	
\end{itemize}


\section{Verzování} \index{verzování}


\subsection{Základní pojmy}
%todo udělat podrobný návod pro VSCode 

{\bf Git} je program, který umí uchovávat jednotlivé verze souborů, zobrazovat rozdíly mezi nimi, slučovat změny více uživatelů a tak dál. Pro vývoj softwaru pro roboty je to naprosto nezbytný nástroj. \index{git}

Podrobnější představení gitu je  \href{http://www.kutac.cz/blog/pocitace-a-internety/jak-na-git-dil-0-co-proc-jak/}{zde}.

{\bf Github}\footnote{\url{www.github.com}}  je v současnosti jeden z nejpoužívanějších webů pro tvorbu a~správu repozitářů.

Návody pro github jsou například zde: 

\begin{itemize}
	\item \href{http://www.kutac.cz/blog/pocitace-a-internety/jak-na-git-dil-1/}{Github na příkazovém řádku v Linuxu.}
	\item \url{https://help.github.com}
	
	% https://help.github.com/articles/adding-an-existing-project-to-github-using-the-command-line/	
	% https://help.github.com/articles/adding-a-remote/
\end{itemize}

 

{\bf Repozitář} \index{repozitář} je skupina souborů nějakého projektu, která navíc obsahuje komentovanou historii všech změn projektu. Používá se pro zálohování programů a sdílení a společné týmové práci na rozsáhlejších projektech, především programátorských a textových. 
 
 Repozitář, ve kterém je i tato dokumentace, je na adrese 
 \url{https://github.com/RoboticsBrno/RobotikaBrno-guides/tree/RoboticsManual}.   

Pro práci s repozitáři je důležitý pojem {\bf commit} \index{commit} -- je to jeden \uv{kus hotové práce}. 
Například naprogramování  nové funkce, přidání kapitoly do textu a podobně. 
Ke každému commitu píšeme při vytvoření anglicky komentář, aby bylo jasné, čeho se daný commit týká.
\index{commit} 

Postup práce je následující: na webu github.com si vytvoříte repozitář. Na svůj počítač si nainstalujete git. Stáhnete k sobě na počítač aktuální verzi repozitáře, upravíte, co potřebujete, vyrobíte commit a upravené soubory nahrajete zpět na server. Podrobněji v návodech níže. 
Neustálému komentovanému ukládání jednotlivých verzí se říká {\bf verzování}. 

Verzování na web github lze provádět z příkazové řádky (terminálu) nebo s~pomocí různých programů, například prostředí \hyperref[vscode]{VSCode} -- viz dále. 



\subsection{Instalace gitu a stažení repozitáře} \label{instal_github}

\begin{enumerate}
	\item Na webu \url{https://github.com} si vytvořte účet a přihláste se. 
	Vpravo můžete založit nový repozitář (new repository) nebo se můžete přepnout do už existujících repozitářů. 
	
	\item Na svém počítači si nainstalujte git. V Linuxu napište do 
	\hyperref[terminal]{terminálu} : {\tt sudo apt-get install git}. 
	Ve Windows stáhnete instalační program z adresy 
	\url{https://git-scm.com/download} a instalujete jako libovolný jiný program. 
	
	\item Vytvořte si lokální složku a stáhněte do ní repozitář.  
	V Linuxu se přepněte do složky, ve které chcete repozitář mít a napište do terminálu: 
	{\tt git clone <cesta\_k\_repozitáři\_na\_webu> }. 
	 Git vytvoří v aktuální složce podsložku s~kopií repozitáře z webu.  
	Cestu potřebnou pro příkaz {\tt clone} zjistíte tak, 
	že se na webu přepnete do požadovaného repozitáře a kliknete na zelené tlačítko \uv{Clone or download}. 
	\item Pokud má repozitář více tzv. větví, přepnete se do požadované větve příkazem 
	{\tt git checkout <nazev\_vetve>}, například {\tt git checkout RoboticsManual}.
\end{enumerate}

\subsection{Jak vytvořit commit a nahrát změny na server pomocí terminálu}

\begin{enumerate}
	\item Změny, které chcete zahrnout do commitu,  musíte do nejprve přidat příkazem {\tt git add <vybrane\_zmeny>}. 
	Příkaz {\tt git add .} přidá všechny nové změny.  Všimněte si, že před tečkou je 
	v příkazu {\tt git add .} mezera. Příkaz je možné libovolně opakovat. 
	
	\item Pokud si nejste jistí, co vše už je nebo není přidáno, použijte příkaz {\tt git status}
	
	\item Příkaz {\tt git commit} vytvoří nový commit ze všech změn přidaných příkazem {\tt add}. Co nepřidáte, to v commitu nebude!
	Protože chceme vědět, co jsme v minulosti dělali, použijeme tvar: 
	{\tt git commit -m ``Tady je napsáno, co tento commit obsahuje.'' }  I commitů můžeme vytvořit více za sebou. Všechny zůstávají na lokálním počítači, pokud je nepošlete na server. 
	
	\item Příkaz {\tt git push} pošle všechny dosud vytvořené commity na server. Vyžaduje přihlášení a heslo. (Nezapomeňte, že v~Linuxu se heslo nezobrazuje, ani ve formě teček). 
	
	
\end{enumerate}



