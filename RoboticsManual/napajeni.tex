\section{Napájení a baterie}


 Napájení robota je obvykle složitější, než by se mohlo zdát. 

Vše na robotovi napájíme stejnosměrným proudem. Pro jeho zajištění používáme různé typy článků baterií.

Všechny tyto články jsou nabíjecí a pro každý typ potřebujeme nabíječku.

Používají se obvykle zapojené do série podle výše potřebného napětí. 


{\bf  Všechny obvody robota (řídící deska, motory, serva, další desky) musí mít společnou zem. }  


Pro řídící desky {\bf musí}  být určené samostatné články, jiné než pro pohony motorů a serv. Důvod je ten, že motory při práci produkují napěťové špičky, které by se přes společné kabely dostaly do čipu a způsobily by, že čip přestane fungovat.  


\subsection{Přehled typů článků}

\begin{description}
	
\item[AAA nabíjecí baterie (mikrotužky)]~\\ \index{baterie!AAA}
Jeden článek má napětí 1,2~V, plně nabitý i 1,4~V. 
Používáme pro pohon malých motorků a malých robotů (3pi). 
Nabíjíme klasickými nabíječkami. 

\item[AA nabíjecí baterie (tužky)]~\\ \index{baterie!AA}
Jeden článek má napětí 1,2~V, plně nabitý i 1,4~V. 
Používáme pro pohon malých motorků a malých robotů, když je na robotovi dost místa na baterie. 
Nabíjíme také klasickými nabíječkami.

\item[Li-On články]~\\ \index{baterie!Li-On} \index{baterie!18650}
Nejčastěji se používají ve velikosti 18650. Jeden článek má asi 3,7~V, plně nabitý i 4,2~V. 
Nabíjíme je například nabíjecí deskou \href{https://laskarduino.cz/napajeni-zdroje/230501-nabijecka-li-ion-clanku-tp4056.html?gclid=EAIaIQobChMI5_DTrqTA3QIVUeR3Ch1GTwayEAQYASABEgJGz_D_BwE}{TP4056}, kterou na jedné straně připojíme k USB portu a na druhé straně pomocí drátků připájíme k držáku baterie. 
Používáme 2 články zapojené \hyperref[seriove]{do série} pro pohon středních robotů a napájení řídících desek, které mají vlastní stabilizaci napětí (RBControl).   

\item[Ni-Cd články]~\\ \index{baterie!Ni-Cd}
Jeden článek má napětí 1,2~V, plně nabitý i 1,4~V.
Mohou dodávat desítky ampér a nelze je snadno zničit velkým odběrem.
Používáme obvykle 6 nebo 10 článků v baterii pro pohon velkých robotů.  

\item[Li-Pol články]~\\ \index{baterie!Li-Pol}
Jejich výhody (nízká hmotnost a malé rozměry) při stavbě robotů obvykle nevyužijeme a kvůli relativně složitému nabíjení, určité nebezpečnosti a snadnému zničení podvybitím je nepoužíváme. 

\item[Power-banky]~\\ 
Poskytují stabilizované napětí 5~V a proto jsou ideální pro napájení řídících desek. 
Nabíjejí se přímo z USB portu. 


\item[USB port]~\\ 
Při stavbě a tréninku často nabíjíme řídící desky přímo z USB portu, který poskytuje stabilizovaných 5~V. 
Podle specifikace má poskytovat 100~mA, ale běžně z něj lze odebírat 500~mV, aniž by mu to vadilo. 
Některé porty zvládnou i 1000~mA, ale to doporučujeme zkoušet pouze tam, kde vám nevadí, že port shoří. 
On by tedy shořet neměl, měl by mít ochranné pojistky, ale jeden nikdy neví ... 

\end{description}

\section{Napětí potřebná pro různé části robota~a doporučené baterie}


Motory potřebují cca 6~--~12~V podle toho, jaký výkon po nich chceme. 
Obvykle se používá 7,2~V pro pohon středních robotů a 12~V pro pohon velkých robotů. 

Servomotory se napájí 5~V, snesou i 6~V, při vyšším napětí nejspíš shoří.  

Pro řídící desky obecně potřebujeme {\bf stabilizované} napětí.   

Rídící desky klonu Adruino potřebují 5~V stabilizovaného napětí. Tomu vyhovuje běžná Power-banka nebo USB port. 


ESP32 a jeho shieldy (ALKS) potřebují 3,3~V stabilizovaného napětí, ale protože mají na sobě stabilizátor z 5~V na 3,3~V, můžeme je také napájet z Power-banky nebo USB portu. 


Pokud bychom chtěli kvůli komunikaci (například přes sériovou linku) propojit desky ESP32 a Arduino, musíme mezi ně zařadit tzv. převodník napěťových úrovní, který zajistí převod signálu z 3,3~V na 5~V a zpět.  


Deska RBControl má na sobě stabilizátor 7805 a tzv. step-downy  pro serva, takže stačí ji připojit na 2 Li-On články zapojené do série, případně na jiný zdroj napětí 8~--~10~V.      %todo doplnit step-down 









