\documentclass[12pt]{report} % report - typ/šablona dokumentu, umožňuje \chapter
\usepackage[czech]{babel} % nastavuje české popisky např. u obsahu, referencí, tabulek, obázků 
\usepackage[utf8]{inputenc} % použito UTF8 kvůli češtině (zvládá prakticky všechny jazyky na světě)

%\usepackage{indentfirst} % odsazuje první odstavec v kapitole

\usepackage{color} % balíček pro obarvování textů
% \color{blue}
\usepackage{xcolor}  % zapne možnost používání barev, mj. pro \definecolor
\definecolor{mygreen}{RGB}{0,150,0} % nastavení barev odkazů 
\definecolor{myblue}{RGB}{0,0,200} 
\definecolor{commentgreen}{RGB}{0,100,0} % nastavení barev pro příklady z C++
\definecolor{deepblue}{rgb}{0,0,0.7}
\definecolor{deepred}{rgb}{0.6,0,0}
\definecolor{deepgreen}{rgb}{0,0.5,0}


\usepackage[T1]{fontenc} % import tučného písma typu tt pro prostředí listings
\usepackage{listings} % balíček pro obarvování syntaxe ukázek programů v textu
%\usepackage{listingsutf8} %nutné pro \usepackage{listings} aby to jelo v UTF-8
\lstset{
		extendedchars 	= false,
		language      	= C++,
		basicstyle      = \ttfamily,
		keywordstyle     = \bfseries,
		%identifierstyle = \color{brown},
		commentstyle    = \color{commentgreen},
		otherkeywords	= {self},             % Add keywords here
		emphstyle		= \color{deepred},    % Custom highlighting style
		stringstyle	 	= \color{deepgreen},		
		keywordstyle	=\color{deepblue},
		stringstyle     = \color{magenta},
	} % písmo a  barvičky  by možná chtěly doladit - nějaký dobrovolník? 

\lstset{literate=    %sem se musí doplnit všechny znaky s diakritikou z kódů se zvýrazněnou syntaxí
	{á}{{a}}1 {š}{\s}1 {ř}{\r}1  } %{š}{\v s}1 -nejede {á}{{\'a}}1 - jede

% \begin{lstlisting} Tady je zdrojový kód např. v C++ \end{lstlisting} - pro UTF-8 NE
%\lstinputlisting{source_filename.py} vloží soubor z daného místa a obarví


\usepackage{hyperref} % balíček pro hypertextové odkazy
% \url{www.odkaz.cz}
% \href{http://www.odkaz.cz}{Text který bude jako odkaz}
%\hyperlink{label}{proklikávací_text} - odkaz na text 
% \hypertarget{label}{cíl_odkazu} - cíl odkazu  


\hypersetup{colorlinks=true, linkcolor=myblue, urlcolor=mygreen, citecolor=blue, anchorcolor = magenta,
	 linktocpage = true, frenchlinks } % nastavení barvy odkazů 
	% bookmarksopen=true, bookmarksnumbered=true, bookmarksopenlevel=1 - nastavuje rozbalování levého menu       
	


\usepackage{makeidx} % slouží pro vytváření indexů/rejstříků
\makeindex % zapíná vytváření/ukládání indexů
% \index{key} % ukládá index
% \printindex % tiskné indexy/rejstřík
% aby se rejstřík vytvořil, je nutné samostatně spustit program makeindex <hlavní soubor projektu> nebo v programu TeXstudio v menu Nástroje položku Rejstřík

\usepackage{graphicx} % pro vkládání obrázků a příkaz "\includegraphics"
%% samotné vložení obrázku
%\begin{figure}
%	\includegraphics[width=\textwidth]{../img/when_use_latex.png}
%	\caption{Kdy se vyplatí použít \LaTeX} % popis, který se zobrazí pod obrázkem
%	\label{moje_navesti} %identifikuje objekt, který lze pak referencovat
%\end{figure}
%% ---------------------------

%% todonotes packagege -> POZOR: nepodařilo se mi je zprovoznit dohromady s htlatex/make4ht (LaTeX -> HTML) (Jarek Páral 2018-02-04)
% alternativa -> \marginpar{Poznámka 1} nebo balíček fixme
%\usepackage{todonotes} % vkládání poznámek
%\usepackage[colorinlistoftodos,prependcaption,textsize=tiny]{todonotes}

\usepackage[author=,status=draft]{fixme} % vkládání poznámek  
% dva módy (status): draft (poznámky se zobrazují v PDF) / final (poznámky se nezobrazují v PDF)
\fxusetheme{color}
%% samotné vložení poznámky - varianty (rozlišené barvou textu poznámky)
%\fxnote{Poznámka} 		% zelená
%\fxwarning{Poznámka}	% oranžová	
%\fxerror{Poznámka}		% červená
%\fxfatal{Poznámka}		% sytě červená
%\fxnote*{Text poznámky}{text v textu, který se má zvýraznit} % potřeba mít * za fxnote
%\listoffixmes			% zobrazí seznam poznámek 
%%


%opening
%\title{}
%\author{}

% Délky a mezery pro celý dokument
\setlength{\parskip}{0.5ex plus 0.5ex minus 0.2ex} %Nastavuje velikost vertikální mezery mezi odstavci:
	% první číslo je mezera mezi odstavci, druhé nevím a třetí je mezera před začátkem nové kapitoly a podkapitoly
\parindent=0pt % odstavce nebudou odsazeny zarážkou (velikost odsazení prvního řádku odstavce)
\usepackage{enumitem} % pro příkaz \setlist
\setlist{itemsep = -1pt, topsep=3pt} % nastavuje mezery mezi a před výčtovým prosředím (enumerate, itemize ...)


%%%%%%%%%%%%%%%%%%%%%%%%%%%%%%%%%%%%%%%%%%%%%%%%%%%%%%%%%%%%%%%%%%%%%%%%
%%%%%%%%%%%%%%%%%%%%%%%%%%%%%%%%%%%%%%%%%%%%%%%%%%%%%%%%%%%%%%%%%%%%%%%%%

\begin{document}

\input prvnistrana.tex

\tableofcontents  \label{obsah}  % zobrazí obsah

\chapter{Začínáme}
	\input zacatek.tex
	
\chapter{Mechanika}
	\input mechanika.tex

\chapter{Elektronika}
	\input soucastky.tex
	\input veliciny.tex
	\input mereni.tex
	\input pajeni.tex
	\input motory.tex
	\input napajeni.tex
	\input sbernice.tex %todo dodělat
	\input osciloskop.tex	
	
\chapter{Software}
	Veškerý zde popisovaný a~doporučovaný software je (minimálně pro vzdělávací účely) freeware.
	\input onshape.tex
	\input vscode.tex
	\input software.tex

\chapter{Programování čipů v C++}
	\input cpp.tex
	\input cppcipy.tex
	\input cpppr.tex
	\input bitybajty.tex

\chapter{Přílohy}  \addcontentsline{toc}{chapter}{Přílohy}
	\input prilohy.tex 
		 
 \addcontentsline{toc}{chapter}{Rejstřík} 

 \printindex   
 \label{rejstrik}
 
%todo doplnit do rejstříku velká počáteční písmena a seřadit ho podle české abecedy 
%todo vytvořit proklikávání v  rejstříku 
%todo software pro videonávody?

%todo doplnit pájení
%todo udělat podrobný návod pro VSCode do gitu
%todo doplnit sekci \section{Příkazy C++ pro čipy}
%todo do kapitoly Onshape doplnit fotky s popisy + \index , taky do osciloskopu
%todo nepájivé kontaktní pole
% stavový automat 
% program vlna 
% Adruino IDE 
% Laser ve Lablabu - ovládání
% Baterie LiON a~jejich nabíjení 
% Převodník napěťových úrovní 
% Sběrnice - USART/UART, IIC, SPI, další?
% propojení čipů s mobilem 
% součástky: tlačítko včetně zapojení, barevný senzor 
% pokud nemáte hodně vývodů na desce, potřebujete nepájivé pole po rozvedení napájení (GND, 5V nebo 3V3). 
% verzování do VSCode 
% jak přesně funguje funkce map 
% otestovat řízení serva a zkontrolovat piny GND a data 
% enkodéry 
% doplnit step-down 

\end{document}

Centrum softwaru pro Ubuntu. Alternativně můžete Centrum spustit příkazem software-center

https://github.com/RoboticsBrno/Robotarna-service/blob/master/install_arduino-esp32.md

nainstalování ESP32 pro Arduino IDE -> problém 
%todo návod do Arduino IDE: http://navody.arduino-shop.cz/navody-k-produktum/vyvojova-deska-esp32.html

