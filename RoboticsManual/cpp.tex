\section{Základní doporučení pro programování} \label{programovani:zaklady} %todo sepsat kapitolu: program v textu, vývojový diagram, stavový automat 

\section{Základy syntaxe v jazyce C}

\subsection{Základní pojmy}

\label{cpp} Upozornění: V~jazyce C a C++ se {\bf rozlišují velká a~malá písmena -- jazyk je case sensitive}.  

\subsubsection{Bity a bajty}

Každá číslice ve \hyperref[dvojkova]{dvojkové} soustavě reprezentuje jeden {\bf bit} \index{bit} (nabývá hodnot 0 nebo 1). 

Osm bitů dohromady tvoří \index{bajt} {\bf bajt}. 
Nejnižší bit v bajtu leží vpravo (tzv. {\bf nultý bit} \index{bit!nultý} \index{nultý~bit} ),  
další je nalevo od něj (první bit), až do sedmého bitu, který je nejvíce vlevo.

\begin{description}

\item[příkaz] \index{příkaz} -- povel pro procesor, říká procesoru, co má dělat.
Příkazy probíhají postupně, nejprve se udělá jeden a pak další, který je na řadě.
Za příkazy dáváme středník (;).

\item[syntaxe]\index{syntaxe} -- \uv{pravopis} programovacího jazyka, říká, jak psát příkazy programovacího jazyka tak, aby nám počítač rozuměl 

\item[preprocesor]\index{preprocesor} -- před samotným překladem programu do \textit{*.hex} souboru proběhnou tzv. direktivy preprocesoru (něco jako příkazy, podrobně \index{direktiva preprocesoru})

%\footnote{něco jako příkazy, podrobně viz např. Kadlec: Programování v~jazyce C, str. 18 }

Příklady: \verb@#include <>@ -- pro vkládání hlavičkových souborů \\
 \verb@#include ""@ -- pro vkládání vlastních hlavičkových souborů \\
 \verb@#define KONSTANTA HODNOTA_KONSTANTY@ -- pro definici konstanty, všude kde se v kódu vyskytne text \verb@KONSTANTA@ bude tento text nahrazen\\  \verb@HODNOTA_KONSTANTY@

\item[proměnná] \index{proměnná}  -- je místo v paměti, kde jsou uložena nějaké data

\item[datový typ]\index{datový typ}\label{data1}  -- říká nám, v jaké podobě jsou data v proměnné uložena. V proměnné jsou pouze jedničky a nuly, pomocí datového typu mikrokontrolér pozná, jestli ta data jsou čísla, nebo znak, v jakém rozsahu jsou čísla, atd...
používané datové typy:   \\
\verb@uint8_t@ -- celá čísla od 0 do 255  \\  
\verb@int8_t@ -- celá čísla od -127 do 127 \\
\verb@uint16_t@ --  celá čísla od 0 do 65535  \\
\verb@int16_t@ -- celá čísla od -32767 do 32767 \\
\verb@uint32_t@ -- celá čísla od 0 do 4294967296 \\
\verb@int32_t@ -- celá čísla od -2147483647 do 2147483647 \\

\item[operandy] -- jsou čísla nebo výrazy, které \uv{vstupují do %výpočetních operací?
	 operace} 

\item[operátory] -- nám říká, jaké operace provedeme s operandy (co s nimi uděláme) 

Příklad: \verb@5 + 2@, přitom \verb@5@ a \verb@2@ jsou operandy a znaménko plus je operátor, který nám říká, že čísla chceme sečíst (provést operaci sečítání)


\item[poznámka] -- cokoliv se objeví v kódu mezi znaky \verb@/*@  a \verb@*/@, tak je poznámka a bude před překladem odstraněno z kódu. 

Pokud chceme zapoznámkovat pouze jeden řádek, použijeme \verb@//@

Příklady:

%\lstinputlisting{priklady_c/poznamky.cpp}

\begin{verbatim}
/* Vše co je napsáno zde
je poznámka a nebude překládáno
*/
\end{verbatim}

\begin{verbatim}
// Toto je poznámka, která může popsat funkci nějakého úseku kódu
\end{verbatim}

\end{description}


\subsection{Výrazy a operátory}

Výraz je něco, co nabývá nějaké hodnoty, např.: \verb@5 + 3@ je výraz, 
který nabývá hodnoty \verb@8@. \verb@5 > 3@ je výraz nabývající hodnoty \verb@true@ neboli pravda, 
\verb@x <= 10@ je výraz který je pravdivý, pokud proměnná \verb@x@(kterou musíme mít deklarovanou) menší nebo rovna číslu \verb@10@. 

\subsubsection{Aritmetické operátory}
\verb@+@ sčítání \\
\verb@-@ odčítaní \\
\verb@*@ násobení \\
\verb@/@ dělení \\

\subsubsection{Operátory inkrementace a dekrementace}
pokud chceme zvýšit hodnotu proměnné o jedničku, napíšeme 
\verb@++název_proměnné@;
pokud chceme snížit hodnotu proměnné o jedničku, napíšeme \verb@--název_proměnné@;

Příklad:
\begin{verbatim}
uint8_t i = 5;  // Vytvoření proměnné i a dosazení hodnoty 5 do ní.  
uint8_t j = 42; // Vytvoření proměnné j a dosazení hodnoty 42 do ní. 

++i;  // Hodnota i se zvýší o jedničku na číslo 6.
--j;  // Hodnota j se sníží o jedničku na číslo 41.
\end{verbatim}


%\subsubsection{Přiřazovací operátor}
%= 

\subsubsection{Logické výrazy a operátory} \label{logika1}

Přehled logických operátorů: \\
\verb@==@ porovnání - rovnost \\ 
\verb@!=@ porovnání - nerovnost \\
\verb@&&@ logický součin \\ 
\verb@||@ logický součet \\
\verb@! @ negace \\ 
\verb@< @ menší než \\
\verb@<=@ menší nebo rovno \\
\verb@> @ větší než \\
\verb@>=@ větší nebo rovno \\


\textbf{Logický výraz}\index{logický výraz}\index{výraz!logický} může nabývat pouze dvou hodnot -- pravda (true) a nepravda (false).  
Obvykle je nepravda reprezentovaná nulou a pravda každým nenulovým číslem. Logické výrazy se používají v podmínkách (viz např. \ref{if1})\\.

  \textbf{Logické operace}\index{logická operace}\index{operace!logická} se používají při vyhodnocování logických výrazů.    

Příklady: \\
\verb@a == b@ výraz je pravdivý, pokud se \verb@a@ rovná \verb@b@ \\
\verb@a != b@ výraz je pravdivý, pokud se \verb@a@ nerovná \verb@b@ 
lze to napsat i konkrétněji, např.: \verb@a != 5@, výraz je pravda pokud se \verb@a@ nerovná \verb@5@, pokud se rovná výsledkem výrazu je nepravda(false)
\verb@b > c@ výraz je pravdivý, pokud je \verb@b@ větší než \verb@c@ \\
\verb@b >= d@ výraz je pravdivý, pokud je \verb@b@ větší nebo rovný \verb@d@ \\

\verb@!(a > b)@ výraz v závorce je pravdivý, pokud je \verb@a@ větší než \verb@b@, 
ale pak je negováno (z pravdy se stává nepravda a naopak), tj. celý výraz je nepravdivý, 
pokud je \verb@a@ větší než \verb@b@, pokud je a menší nebo rovno, tak je výraz pravdivý. \\

\verb@!(a == b)@ výraz \verb@(a == b)@ je negován znaménkem \verb@!@, to znamená, že výraz je pravdivý, 
pokud se výraz \verb@a@ nerovná výrazu \verb@b@, v podstatě se to dá napsat i takto: \verb@a != b@ \\
Pozor: negace výrazu \verb@(a > b)@, tj. \verb@!(a > b)@ není to samé jako výraz \verb@(a < b)@, 
ale správně je to \verb@(a <= b)@ \footnote{Platí zde určité zákonitosti viz De~Morganovy zákony} \\

Logický součin \verb@&&@ se používá, pokud budeme potřebovat spojit dva nebo více výrazů dohromady,
 např.: \verb@(a > b)&&(c == d)@, výsledkem tohoto výrazu bude logický součin výrazů v závorkách,
  pro logický součin platí, že je pravda pokud oba výrazy jsou pravdivé, jinak je výsledek nepravda, 
  tj. zde bude pravda pouze pokud bude \verb@a@ větší než \verb@b@ a zároveň bude platit, že \verb@c@ se rovná \verb@d@.
Logický součin použijeme, pokud musí všechny výrazy být pravda.

Logický součet \verb@||@ je pravdivý, pokud alespoň jeden výraz je pravdivý. 
Např.: \verb@(e <= f)||(g != 3)@ výraz bude pravda, pokud bude platit, že \verb@e@ je menší
 nebo roven \verb@f@, nebo bude platit, že \verb@g@ se nerovná \verb@3@, anebo klidně budou platit oba výrazy.


\subsection{Nejčastější příkazy C++}

\subsubsection{Přiřazovací příkaz} 
Použijeme, pokud chceme, aby se do proměnné uložila nějaká data. 

Syntaxe: \\
\verb@datový_typ název_proměnné@; \\ 

syntaxe vkládání: \\ 
\verb@název_proměnné = hodnota_která_se_má_uložit@;

Data do proměnné můžeme vložit rovnou při vytváření proměnných:

  \verb@datový_typ název_proměnné = hodnota@;

 Příklad: \verb@uint8_t b = 5@;  Vytvoří se proměnná pojmenovaná \verb@b@ (která je v rozsahu od 0 do 255) a uloží se do ní číslo \verb@5@. \\
\verb@int16_t B;@  Vytvoří se proměnná \verb@B@. \verb@B@ není to samé jako \verb@b@, protože jazyk C rozlišuje VELKÁ a malá písmena. \\
\verb@B = 1024;@  Do proměnné B se uloží hodnota 1024;

\subsubsection{Blok příkazů} 
Pokud chceme někam dát více příkazů, ale můžeme tam dát pouze jeden příkaz, tak je dáme do složených závorek
Příklad: \\ 
\begin{verbatim}
{
  PŘÍKAZ1;
  PŘÍKAZ2;
  PŘÍKAZ3;
}
\end{verbatim}



\subsubsection{Podmiňovací příkaz (\texttt{if})} \label{if1}

Použijeme, pokud chceme, aby se program mohl rozhodnout na základě nějaké podmínky. \\ 
syntaxe:

\begin{verbatim}
if(PODMÍNKA)
  PŘÍKAZ1;
else
  PŘÍKAZ2;
\end{verbatim}

Pokud platí \verb@PODMÍNKA@ v kulaté závorce, vykoná se \verb@PŘÍKAZ1@, pokud neplatí, vykoná se \verb@PŘÍKAZ2@.
Za \verb@if@ nebo \verb@else@ může být pouze jeden příkaz, pokud jich tam chceme dát více, použijeme blok.
Větev \verb@else@ je nepovinná.

Příklad:
\begin{verbatim}
if(a > 5) // Pokud platí, že 
          //proměnná a je větší něž 5, tak se
{
  b = a;  // do proměnné b uloží hodnota, která je v proměnné a
  c = 2;  // do proměnné c se uloží číslo 2.
} // Pokud podmínka neplatí, tak se nevykoná nic.
\end{verbatim}

podmínka může být výraz např.: 
\begin{verbatim}
a > 5 // podmínka bude platit, 
      // pokud proměnná a je větší než 5
\end{verbatim}
podmíněné příkazy lze vnořovat - v bloku příkazů může být další if
Příklad:
\begin{verbatim}
if(PODMÍNKA1)
{
  PŘÍKAZ1;
  if(PODMÍNKA2)
  {
    PŘÍKAZ2;
  }
  PŘÍKAZ3;
}
else 
{
  PŘÍKAZ4;
}
\end{verbatim}

Pokud platí \verb@PODMÍNKA1@ tak se vykoná \verb@PŘÍKAZ1@, pak se zkontroluje \verb@PODMÍNKA2@,
 pokud platí, tak se vykoná \verb@PŘÍKAZ2@, pokud ne, program pokračuje dál a vykoná \verb@PŘÍKAZ3@.
V případě, že neplatí ani \verb@PODMÍNKA1@, tak se blok příkazů přeskočí a vykoná se blok příkazů za \verb@else@, tj. \verb@PŘÍKAZ4@.

Je tu možnost také do větve \verb@else@ napsat daší \verb@if else@, vypadá to takto:
\begin{verbatim}
if(PODMÍNKA1)
{
  PRÍKAZ1;
}
else
{
  if(PODMÍNKA2)
  {
    PRÍKAZ2;
  }
  else
  {
    PRÍKAZ3;
  }
}
\end{verbatim}
Pokud platí \verb@PODMÍNKA1@, tak se vykoná \verb@PŘÍKAZ1@, pokud ne, tak se zkontroluje
 \verb@PODMÍNKA2@, pokud platí tak se vykoná \verb@PŘÍKAZ2@, pokud neplatí, tak se vykoná
  \verb@PŘÍKAZ3@. Výše zapsaný kód sice bude fungovat, ale není přehledný, a proto z důvodu lepší orientace v kódu zvolíme tento zápis:
\begin{verbatim}
if(PODMÍNKA1)
{
  PRÍKAZ1;
}
else if(PODMÍNKA2)
{
  PRÍKAZ2;
}
else
{
  PRÍKAZ3;
}
\end{verbatim}

\subsubsection{Příkaz několikanásobného větvení (\texttt{switch})}
Pokud by takto zapsaných podmínek bylo moc a vztahovali by se k jedné řídící 
proměnné lze použít příkaz několikanásobného větvění neboli switch. Tak například místo zápisu:
\begin{verbatim}
if(a == 1)
{
  PŘÍKAZ1;
}
else if(a == 2)
{
  PŘÍKAZ2;
}
else if(a == 3)
{
  PŘÍKAZ3;
}
else if(a == 4)
{
  PŘÍKAZ4;
}
else
{
  PŘÍKAZ5;
}
\end{verbatim}
Lze napsat:
\begin{verbatim}
switch(a)
{
  case 1:
    PŘÍKAZ1;
    break;
  case 2:
    PŘÍKAZ2;
    break;
  case 3:
    PŘÍKAZ3;
    break;
  case 4:
    PŘÍKAZ4;
    break;
  default:
    PŘÍKAZ5;
    break;
}
\end{verbatim}
Čím se to celé zjednodušuje.

\subsubsection{Cyklus \texttt{for}  (cyklus řízený proměnou )} 
Tento cyklu použijeme, pokud předem známe kolikrát se má opakovat určitá operace
Syntaxe:
\begin{verbatim}
for(řidící_proměnná; podmínka; zvýšení_nebo_snížení_hodnoty_proměnné )
{
  PŘÍKAZ1;
  PŘÍKAZ2;
}
\end{verbatim}

Příklad:
\begin{verbatim}
for(uint8_t i=0; i<4; i++)
{
  rozsvit_ledku();
  pockej_sekundu();
  zhasni ledku();
}
\end{verbatim}
Co tento kód udělá? Dejme tomu, že máme už předem vytvořené funkce \verb@rozsvit_ledku()@, atd... 
Potom cyklus udělá následující: vytvoří proměnnou \verb@i@ datového typu \verb@uint8_t@ (celá čísla od \verb@0@ do \verb@255@),
 uloží do ní číslo \verb@0@. 
 Potom zkontroluje podmínku, zda je proměnná \verb@i@ menší než \verb@4@, pak se provedou příkazy či funkce v těle cyklu,
  tj. rozsvítí se LED-dioda, bude svítit jednu sekundu, a pak zhasne. 
  Pak se provede příkaz \verb@i++@, ten zvýší hodnotu proměnné o jednu, 
  tj. na číslo \verb@1@. Potom se kontroluje znovu podmínka, zdali je
   proměnná \verb@i@ (která má teď hodnotu \verb@1@) menší než \verb@4@, atd... Celkem se tělo cyklu vykoná čtyřikrát.

Cyklus \verb@for@ lze použít i bez proměnné, podmínky a zvýšení hodnoty, potom bude probíhat donekonečna.
Příklad:
\verb@for(;;)@
\begin{verbatim}
{
  rozsvit_ledku();
  pockej_sekundu();
  zhasni ledku();
}
\end{verbatim} 
Tento cyklus bude dělat to samé, co předchozí, s tím rozdílem, že to bude dělat donekonečna a nebude se ptát na podmínku. 

Použití v programu pro robota: Nekonečný cyklus, ve kterém se kontrolují podmínky, zda je na čáře, zda je před ním soupeř, zda má udělat to či ono.


\subsubsection{Cyklus  \texttt{while}  (cyklus s podmínkou na začátku )} 

Použijeme, pokud nebudeme vědět kolikrát mají proběhnout příkazy a funkce v těle cyklu.
Syntaxe:

\begin{verbatim}
while(podmínka)
{
  PŘÍKAZ1;
  ...
}
\end{verbatim}


Příklad:
\begin{verbatim}
while(!je_tlacitko_stiskle())
{
  ujed_1_cm();
}
\end{verbatim}

Tento cyklus bude stát, dokud nezmáčkneme tlačítko. Vyžaduje mít předem naprogramovanou funkci \verb@ujed_1_cm()@ a \verb@je_tlacitko_stiskle()@.

 
%%%%%%%%%%%%%% Napsat cyklus do-while 
  
\subsection{Funkce a procedury} 

 Pokud se nám v zdrojovém kódu opakují dokola stejné příkazy, můžeme vytvořit funkce nebo procedury, které potom voláme a které nám usnadní kód.
syntaxe:
\begin{verbatim}
datový_typ_který_funkce_vrací název_funkce(datový_typ_parametru1 
parametr1, datový_typ_parametru2 parametr2) 
// Parametry jsou nepovinné
{
  PŘÍKAZ1; // Nepovinné příkazy
  PŘÍKAZ2;
  return HODNOTA_KTEROU_FUNKCE_VRACI; 
  // Za return se napíše výraz, který funkce vrací.
}
\end{verbatim}

Příklad:
\begin{verbatim}
// Funkce na sčítání dvou čísel, každé od 0 do 255.
uint16_t umocni(uint8_t zakl, uint8_t exp)
{
  uint16_t vys = 1;
  if(exp > 0)
  {
    for(uint8_t i = 0; i < exp; i++)
    {
      vys = vys * zakl; 
    }
  }
  return vys;
}
\end{verbatim}

pokud někde v programu použiji funkci \verb@secti(10, 2)@, tak výsledkem bude, jako bych napsal \verb@10*10@, tj. \verb@100@.
Příklad:
\begin{verbatim}
c = umocni(a, b); // Do proměnné c se uloží b-tá mocnina čísla a. 
\end{verbatim}

procedura -- je vlastně funkce, jen s tím rozdílem, že procedura má datový typ \verb@void@ % co to je???? 
a nemá \verb@return@ (tj. nevrací žádnou hodnotu).

Příklad:
\begin{verbatim}
void rozsvitLedku()
{
  DDRB |= (1<<PB0); /* Co dělá tento příkaz nemusíte zatím řešit,
   pak se k tomu dostaneme.
   */
}
\end{verbatim}

Pokud někde v programu napíšu \verb@rozsvitLedku()@, tak se rozsvítí LEDka na robotovi.

tělo funkce main -- všechny příkazy, které jsou mezi složenými závorkami, se po zahajení programu postupně vykonají

\label{knihovna}\subsection{Knihovny}

\subsubsection{Vlastní knihovny} prog:knihovna
Z funkcí a procedur se mohou vytvářet soubory pro usnadnění práce. Těmto souborům se říká \textbf{knihovny} \index{knihovna}. 
  Soubory mají koncovku \textit{.h}
Příklad: \verb@avr/io.h@
\verb@moje_knihovna.h@

Vlastní knihovnu poté mohu vložit direktivou \verb@#include@
Příklad:
\begin{verbatim}
#include "moje_knihovna.h"
\end{verbatim} 

Příklad takové knihovny je v kapitole \ref{prog:knihovna} na straně \pageref{prog:knihovna}.

\subsubsection{Použité knihovny}  

Pro práci s mikrokontroléry budeme potřebovat některé už vytvořené knihovny.
Příklad: 
%\verb@avr/io.h@  -- pro základní vstup a výstup
%\verb@avr/interrupt.h@ -- pro práci s přerušením
Knihovna \verb@Arduino.h@ zavádí příkazy pro práci s čipy ATMega~a ESP32.
Knihovna \verb@Learningkit.h@  rozšiřuje možnosti předchozí knihovny~o pohodlnou práci s deskou ALKS.


%\verb@stdint.h@  -- zde jsou definovány datové typy (viz \ref{data1}) 
 
 %%%%%%%%%%%%%%%%%%%%%%%%%%%%%%%%%%%%%%%%%%%%%%%%%%%%%%%%%%%%%%%%%%%%%%%%%%%

