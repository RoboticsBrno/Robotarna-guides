\section{Sběrnice}


\textbf{Sběrnice} \index{sběrnice} neboli \textbf{(komunikační) rozhraní} jsou domluvené postupy/systémy, jak se dva čipy nebo dvě různá zařízení dorozumívají mezi sebou. 
Sběrnice jsou různých typů, pro naše účely stačí znát \index{UART} 
\href{https://maly.gitbooks.io/hradla-volty-jednocipy/23_seriova_komunikace/231_seriova_sbernice_spi.html}{SPI}, 
\href{https://maly.gitbooks.io/hradla-volty-jednocipy/23_seriova_komunikace/232_seriova_sbernice_i2c.html}{I2C}  
a \href{https://maly.gitbooks.io/hradla-volty-jednocipy/23_seriova_komunikace/235_rs-232,_uart,_serial.html}{USART/UART}

\subsection{USART/UART} \label{uart}

Sběrnice UART je nejjednodušší pro propojení dvou zařízení. 
Obě zařízení musí mít společnou zem. 
Pro přenos se používají piny \texttt{Rx} a \texttt{Tx}. 
Přitom platí, že \texttt{Rx} prvního zařízení se musí připojit na \texttt{Tx} druhého zařízení a obráceně.  
Na začátku se musí na obou zařízeních nastavit stejná rychost přenosu a další parametry: tzv. parita, start bit a stop bit. 
Pokud propojujete dvě Arduino desky nebo Arduino a ESP32, mají výchozí nastavení stejné. 
Příklad programu pro UART je v \cite[strana~144]{ard}. 
Další příklad je v kapitole \ref{prog:lorris}.
Pozor, pokud propojujeme zařízení na 5~V se zařízením na 3,3~V, musíme použít 
 \hyperref[prevodnik]{Převodník napěťových úrovní}.

\subsection{I2C} \label{i2c} \label{spi} \index{I2C} \index{SPI}

Rozhraní \textbf{I2C}  umí pomocí dvou pinů (a společné země) připojit k čipu až 127 zařízení. Komunikovat s čipem (tzv. master) může vždy pouze jedno zařízení (tzv. slave), ostatní \uv{poslouchají} na lince, až s nimi čip zahájí komunikaci. 
Sběrnice I2C se neosvědčila tam, kde je větší vzdálenost mezi zařízeními než desítky cm, ideální je, když jsou všechna komunikující zařízení na jedné (řídící) desce. 
Pro větší vzdálenosti mezi zařízeními je vhodná sběrnice \textbf{SPI}. 
Příklad programu pro I2C je v \cite[strana~152]{ard}.

%\subsection{SPI}

