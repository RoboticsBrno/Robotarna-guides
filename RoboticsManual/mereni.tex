\section{Měření základních veličin}

Napětí, proud a odpor měříme pomocí \textbf{multimetru}. \index{multimetr}

\subsubsection*{Zdířky}

\begin{description}
	\item[zdířka COM:] Záporný pól neboli zem (GND), používá se vždy, připojuje se do něj vždy \textbf{černý} kabel.
	 Ostatní zdířky jsou kladné póly a používá se z nich ta, která je potřeba. Připojuje se do nich vždy \textbf{červený} kabel. 
	\item[zdířka V/$ \Omega $:] pro měření napětí a odporu
	\item[zdířka A:] pro měření proudu
	\item[zdířka 20~A]  pro měření velkých proudů (bez pojistky!) nebudeme používat
\end{description}


\subsubsection*{Měření odporu} \index{Měření!odporu}

Multimetr přepněte na rozsahy nahoře, skupina $ \Omega $. 

Po přepnutí na $ \Omega $ se vlevo zobrazí 1. 
To znamená, že měřená hodnota je mimo nastavený měřící rozsah.
V našem případě je to odpor vzduchu mezi měřícími hroty.

Změřte odpor   \hyperref[rezistor]{rezistorů} a) držených v ruce, b) umístěných v nepájivém kontaktním poli nebo položených na lavici.
Pokud se naměřené hodnoty liší, pokuste se vysvětlit proč. 

Při zapojování součástek do nepájivého kontaktního pole dejte pozor, abyste je nezkratovali.

Měřit začněte od největšího očekávaného rozsahu a postupujte dolů.

Pozor, pokud budete měřit odpor rezistorů zapojených se zdrojem v obvodu, 
může se stát, že naměříte vnitřní odpor zdroje. 

Lidské tělo má taky konečný odpor -- pokuste se ho změřit.


\subsubsection*{Měření napětí} \index{Měření!napětí}

{\tt DC}  je zkratka pro \textit{direct current } -- stejnosměrný proud. 
{\tt AC} je podobně zkratka pro \textit{alternate current} -- střídavý proud.
Na  multimetru se používají zkratky {\tt DCV}  pro stejnosměrné napětí a {\tt DCA} pro stejnosměrný proud. 
Dále zkratky pro střídavé napětí {\tt ACV} a střídavý proud {\tt ACA}. 
Protože baterie poskytuje vždy stejnosměrné napětí, budeme měřit stejnosměrné napětí a stejnosměrný proud.

Nastavte vhodný rozsah multimetru a změřte napětí na baterii. 

Dále změřte napětí na rezistoru.   

\textbf{Pozor}, pokud měříte \textbf{napětí}, připojte multimetr \hyperref[seriove]{paralelně} (multimetr je mimo měřený obvod a má nastavený obrovský vnitřní odpor).

Pokud měříte \textbf{proudy} (viz dále): připojte multimetr \hyperref[seriove]{sériově}  (multimetr je zapojený do obvodu tak, aby proud tekl přes něj).

%todo +obrázky

Zapojte sériově dva rezistory 10k$ \Omega $ a 20k$ \Omega $ a baterii a změřte napětí na každém 
rezistoru a na baterii, výsledky zapište. Mimochodem, právě jste sestavili tzv. \textbf{napěťový dělič} -- zapojení, které se používá poměrně často.
\index{napěťový dělič}   

Pokuste se získaný výsledek zobecnit a odvodit z Ohmova zákona.


\subsubsection*{Měření proudu} \index{Měření!proudu}

\textbf{Pozor !} Přepojte červený měřící kabel do zdířky A.

Změřte proud tekoucí zapojeným obvodem.

Odpovídá naměřená hodnota očekávání? Ověřte podle Ohmova zákona.


\textbf{Pozor !} Pokud se baterie silně zahřívá, je zkratovaná a musí se okamžitě vypojit z obvodu.

Měřáky na konci měření vypínejte -- šetří se tím podstatně baterie.

% ------------------------------------------------------------------------------------

%Metodické poznámky:

%Na začátku prvního měření se měřáky podepíšou a dál si berou vždy ty, co poprvé.
%Vyučující si zapíše, kdo měří se kterým měřákem.

%Měřáky jsou drahé - chovejte se k nim podle toho.






