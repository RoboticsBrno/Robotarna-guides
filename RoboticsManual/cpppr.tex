

\section{Příklady programů v C++}   \label{cpppr}

Ve všech příkladech níže je uveden vždy obsah souboru {\it main.cpp}. 
Text předpokládá, že nad příklady budete samostatně přemýšlet a~učit se z~nich, proto se to, co bylo řečeno u~prvního příkladu, už neopakuje u~druhého.  
Doporučuji projít soubory {\it LearningKit.h}
a~{\it LearningKit.cpp} (viz v~\hyperlink{explorer}{Exploreru}
adresář {\it .piolibdeps/ArduinoLearningKitStarter\_ID1745/src} ), protože jsou v~nich zkratky typu {\tt L\_R} a~jejich přiřazení pinům čipu.
Další příklady jsou na \url{http://wall.robotikabrno.cz} a~\url{https://www.arduino.cc/reference/en/}.  

\label{cpppr1} \subsection{Blikání LED} 

Program bliká červnenou LED.   

\lstinputlisting{priklady_c/blikani_LED1.cpp}


\label{cpppr2} \subsection{LED zapínaná tlačítkem} 

Žlutá LED zapínaná tlačítkem.  

\lstinputlisting{priklady_c/zapnuti_LED1.cpp}


\label{cpppr3} \subsection{Nejjednodušší PWM} 

PWM \index{PWM} umožňuje (ve spolupráci
s~drivery ) řídit motory, serva a~podobně. Zde je použito na stmívání LED pomocí potenciometru.  %todo doplnit vazby

\lstinputlisting{priklady_c/PWM_01.cpp}


%je potřeba setupLeds(); ?? 


%// ledcSetup(channel, freq, resolution)
%    // channel = 0 - 15
%    // resolution = 10 => 2^10 => 1024
tento kód funguje pro čip ESP32. Pro čipy řady ATMega, které jsou na deskách Arduino uno a~Arduino nano, je potřeba použít tento kód: 
 
%todo dodělat program PWM
Funguje stejně, ale místo příkazu ledcWrite je použit:  
%%%%%%%%%%%%%%%%%%%%%%%%%%%%%%%%%%%%%%%%%%%%%%%%%

\subsection{Infrasenzor na čáru} \label{prog:qrd1114}

Použijeme součástku \hyperref[qrd1114]{QRD1114}, kolektor (C) je připojený na A0. Zdrojový kód je pro Arduino Uno. 

\lstinputlisting{priklady_c/qrd1114.cpp}