\section*{Příloha A~-- Hodnoty vybraných součástek}


\hypertarget{1N4148}{}
{\bf Dioda 1N4148}  \index{dioda!1N4148}\index{1N4148!dioda}

Maximální napětí: 100 V

Maximální proud: 200 mA

Maximální výkon:  500 mW

Úbytek napětí: 1,8 V

\vskip 2mm

\hypertarget{1N4007}{}
{\bf Dioda 1N4007} \index{dioda!1N4007}\index{1N4007!dioda}

Maximální napětí: 1000 V

Maximální proud: 1 A

Maximální výkon: 3 W

Úbytek napětí: 1,1 V

\vskip 2mm

\hypertarget{BCC337}{}
{\bf tranzistor BC337} \index{tranzistor!BCC337}\index{BCC337!tranzistor}

Max. napětí mezi kol. a~emit. $V \rm _{CEO}$:50 V

Max. napětí mezi bází a~emit. $V \rm _{CBO}$: 5 V

Max. proud tekoucí kolektorem $I \rm _C$: 800 mA

Maximální výkon $ P \rm _C$: 625 mW

Zesílení $ h \rm _{fe}$: 100 až 600

\vskip 2mm

\hypertarget{BCC547}{}
{\bf tranzistor BC547} \index{tranzistor!BCC547}\index{BCC547!tranzistor}

Max. napětí mezi kol. a~emit. $ V \rm_{CEO}$: 45 V

Max. napětí mezi bází a~emit. $ V \rm _{CBO}$: 6 V
 
Max. proud tekoucí kolektorem $ I \rm _C$: 100 mA

Maximální výkon $P \rm _C$: 500 mW

Zesílení $h \rm _{fe}$: 110 až 800

\vskip 2mm

\hypertarget{BD911}{}
{\bf tranzistor BD911}  \index{tranzistor!BD911}\index{BD911!tranzistor}

Max. napětí mezi kol. a~emit.$V \rm _{CEO}$: 100 V

Max. napětí mezi bází a~emit. $V \rm _{CBO}$: 5 V

Max. proud tekoucí kolektorem $I \rm _C$: 15 A

Maximální výkon $P \rm _C$: 90 W

Zesílení $h \rm _{fe}$: 5 až 250

Důležité je si všimnout, že minimální zesílení\footnote{Platí pokud bude kolektorem protékat 10~A.}  je 5. 
To znamená, že pokud budeme chtít tranzistorem spínat proud 10~A, tak budeme muset nechat bází téct řídící proud 2~A,
 to znamená, že ho nemůžeme naplno využít, pokud ho budeme řídit mikrokontrolérem, tj. musíme ho spínat jiným tranzistorem. 
 Anebo přijdeme na myšlenku, že pokud budeme chtít řídit velké proudy, budeme potřebovat tranzistory typu MOS-FET, např.: IRF520, IRL3803.

%Maximální proud tekoucí kolektorem $I_C$\footnote{Ang.: Collector Current}

%Maximální výkon $P_C$\footnote{Ang.: Collector Power Dissipation}

%Zesílení $h_{fe}$\footnote{Ang.: DC Current Gain}

\section*{Příloha B -- Poznámky a~vize}



\subsection*{Přerušení}

Co je přerušení? Procesor může zvládnout pouze jednu operaci na jeden tik krystalu, 
postupuje od jednoho příkazu k~druhému a~nemůže jen tak všeho nechat a~věnovat se něčemu jinému, občas je to ale potřeba.
 Při přerušení procesor všeho nechá a~bude se věnovat přerušení, potom co skončí se bude věnovat dál programu tam, kde přestal. 

\subsection*{Baterie}

Dobíjecí: napětí 1,2 V~\index{baterie!dobíjecí}\index{dobíjecí!baterie}

Jednorázové: napětí 1,5 V~\index{baterie!jednorázové}\index{jednorázové!baterie}
 
\subsection*{Vize -- co přidat do textu}

Adruino IDE 

Laser ve Lablabu 

Osciloskop

Baterie LiPol a~jejich nabíjení 

Převodník napěťových úrovní 

Pájení 

Sběrnice - USART/UART, IIC, SPI, další?

% - modrozub

% - propojení s mobilem 

% - verze jako poslední příloha 
% součástky: tlačítko včetně zapojení, barevný senzor 
%seznam literatury 
% jak přesně funguje funkce map 
% otestovat řízení serva a zkontrolovat piny GND a data 


