%todo  Bobčík: 1. lekce pro pokročilé - github > zkontrolovat na 225


\section{Úvod}

\label{konstrukce}
\index{konstrukce~robota} \index{robot!konstrukce}

Roboty stavíme nejčastěji pro nějakou soutěž. 
Podle zadání soutěže se rozhodujete, co bude robot na hřišti dělat, jak se bude orientovat atd.
Z toho potom plynou požadavky na konstrukci. 

Roboty rozlišujeme podle způsobu ovládání a~podle velikosti. 

\subsubsection*{Ovládání robotů} 

Každý robot může být buď řízený nebo autonomní. \label{autonomni}
{\bf Autonomní }znamená, že je naprogramovaný a~během soutěže nebo prezentace se pohybuje samostatně.  \index{robot!autonomní}

Roboty můžeme řídit po kabelu nebo bezdrátově, například přes  \hyperlink{bluetooth}{bluetooth}.   


\subsubsection*{Velikost robotů} 

Na škole a~v~Robotárně stavíme a~programujeme hlavně roboty dvojího typu :\uv{střední} a~\uv{velké}.   
Ostrá hranice mezi nimi není, doporučení pro oba typy se dají kombinovat podle situace.  
Střední je robot se základnou přibližně od 10x10~cm do 25x25~cm.

\section{Mechanická konstrukce robota -- doporučení}

Celého robota nejprve navrhneme ve vhodném \index{CAD} \label{cad} CAD programu:  \href{https://www.onshape.com/}{Onshape}, 
 \href{http://www.solidworks.com}{Solidworks} nebo \href{https://www.autodesk.com/products/fusion-360/students-teachers-educators}{Fusion}.

\index{Solidworks} \index{onshape}
Pokud jako základní materiál zvolíme překližku nebo například plexisklo (kombinované s~díly z~lega),
můžeme díly vymodelované v~CADu nechat vyřezat na laseru na pracovišti \href{https://www.fablabbrno.cz}{Fablab}.
To dramaticky urychluje práci.    
Pro konstrukci prototypů se také hodí měkčené PVC.  

Při stavbě robotů se držíme osvědčeného schématu: 

Podvozek středních robotů tvoří základní deska z~překližky, plexiskla nebo PVC. 
Podvozek velkých robotů tvoří rám z~hliníkových profilů typu \uv{L} tvaru obdélníku, osmiúhelníku nebo kruhu. 

Na podvozku jsou  připevněné dva motory včetně převodovek, které se koupí hotové. 
V~žádném případě se nepokoušejte koupit pouze motor (např. protože je levnější) a~vyrábět si převody sami.  
U velkých robotů volíme motory z~akušroubováků nebo z~akuvrtaček, jiné typy motorů jsou buď pomalé nebo slabé.
U středních robotů stačí motory modelářské, dnes nejčastěji kupované z Číny.

Dále jsou zde dvě kola, každé připojené ke svému motoru a~jedna nebo více podpěr podvozku, 
obvykle z~kartáčku na zuby.
Motory s~koly lze umístit doprostřed nebo dozadu, podle toho, co má robot na hřišti dělat. 

Na podvozek se umisťuje konstrukce z~hliníkových tyčí, profilů nebo z~merkuru, s~pomocí které robot plní svoje úkoly. 
U velkých robotů se méně tuhé materiály neosvědčily, u středních často stačí konstrukce z překližky.     