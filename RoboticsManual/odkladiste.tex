
 \section{Věci okolo programování v C}
  
  není rovno se píše != (ne <>)
  
  řetězcové pole se píše proměnná[] při deklaraci, proměnná (bez závorek) potom 
  
 znak dělení /   pokud jsou oba operandy celočíselné, tak se provede CELOČÍSELNÉ dělení, pokud je alespoň jeden operand desetinný, pak se provede normální dělení
 
 \verb#&&# - znak pro and; ostatní logické operátory  mají taky svoje znaky 
 
 \verb#||# -- znak pro or;
 
 číslo a znak je to samé 
 
 řetězec je pole znaků - znaky do něj musím vkládat na danou pozici 
 délka řetězce je daná při deklaraci vyjímka je přiřazení typu   
 
 apostrofy 'r' označují jeden znak, "x" uvozovky označují řetězec - to je rozdíl 
 
 při zadávání konstant ta konstanta nesmí začínat nulou, jinak ji bere, že je v osmičkové soustavě - a výsledek je chyba, která se moc blbě hledá
 
poslední znak řetězce je 0, je tam navíc, takže délka desetiznakového řetězce je 11 znaků 

všechny registy jsou volatile -- program nemůže v podmínce while ignorovat čtení registrů, i když ta podmínka nic potom nedělá, protože hodnoty v registru se můžou změnit  
-----------------------------------------------------------

\section{USART}


Sběrnice -- umožňuje posílat data mezi různými zařízeními.   

\subsection{Typy sběrnic}

USART - minimum drátů, rychý, jednoduchý, dvě zařízení 

SPI - libovolný počet zařízení, jednoduchá adresace, pouze synchoní přenos 

IIC - víc zařízení v rámci jednoho malého systému, dva dráty, hodiny a data 
v jednom systému může být více masters 
 
\subsection{Výbět typu přenosu }


přenos devíti bitů - devátý bit rozlišuje, jestli je to adresa, nebo data. 
Používá se pro multiprocesorovou komunikaci -- nepoužijeme to  

? Multi-processor Communication Mode str 163 - nepoužívat 

jeden nebo dva stop bity - jeden stačí, kdyby byly dva, může se to hodit pro 
1) delší čas na zpracování signálu, 
2) líp se chytá konec bajtu -- lepší v případě poruchových přenosů 
> budeme používat jeden, nemusí se to nastavovat 


 Parita - kontrolní proces - zahlásí chyby v přenosu - str 169 při zapojení na jedné desce a v blízkém okolí není potřeba, data se neztrácí 
 
 zvolit 38.4 kb/s - je to osvědčená rychlost, malá chybovost a Junior na ní komunikuje - budeme kompatibilní 
 
 synchronní komunikace -- jsou potřeba hodiny (hodinový signál na pinu SCK) 
 hodí se pro dlouhé kusy dat a hlavně pro komunikaci s periferiemi, které jinou možnost neposkytují 
 
 asynchronní komunikace -- stačí propojit Rx a Tx na obou stranách a stejně nastavit oba čipy. není nutné řešit hodiny --  doporučeno   
 
 je to taky jediný způsob, jak komunikovat s počítačem, je to úplně stejný protokol jako RS 232 (sériový port) a liší se to pouze napěťovou úrovní -- MAX232 pouze převádí napěťové úrovně a nic jiného (vlastně je MAX232 DC/DC měnič)
 
 data overrun? 
 
Chceme: asynchronní přenos na 38.4 kb/s bez kontrol  


\subsection{Nastavení}

(žádná knihovna pro usart není)
Je potřeba nastavit: 
- spustit USART (inicializace):  (str 152)
   - nastavit frekvenci přenosu (baud rate)  --- 38.4 kb/s 
   - nastavit, že přenášíme 8-mi bitové znaky bez kontrol  (frame format ) (UCSRB,C)
   - nastavit piny 14 a 15 jako Rx a Tx 
   - že je to  asynchronní přenos (mode of operation), je nastaveno implicitně 
   - pokud chceme používat přerušení od usart, musí být během inicializace 
     globální přerušení vypnuto 
   - přitom musí být URSEL = 1 v registru UCSRC (implicitně je)
   
   
  
  -----------------------------------------------------
  
   
 


 
 



 

 

